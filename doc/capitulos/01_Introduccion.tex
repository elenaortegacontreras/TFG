\chapter{Introducción}
\section{Descripción y contexto} / \section{Descripción del problema}

Según su definición en la RAE la \cite{economía} es la 
"Administración eficaz y razonable de los bienes", dicho esto y teniendo en 
cuenta que nos concierne a todos, es crucial tener control sobre ella. 

En una época en la todo está muy caro, pero a la vez el consumo parece estar 
desvocado, es importante consumir de manera responsable nuestro dinero. Esto 
implica planificación y control sobre cómo y dónde lo gastamos.

Sin embargo, no es fácil hacer el seguimiento de nuestras finanzas. Podemos tener 
varias fuentes de ingresos y/o gastos, por lo que resulta tedioso 
llevar las cuentas al día para analizar adecuadamente nuestro consumo. 

En el ámbito de la economía personal, la digitalización ha supuesto (entre 
otros) la creación de aplicaciones que nos ayudan a gestionar nuestros 
gastos, ahorros, inversiones, etc.

La transición hacia lo digital está cambiando nuestra forma de interactuar con 
la información, nos permite acceso frecuente, práctico y sencillo a ella.
Al agrupar todo en un solo dispositivo que consultamos en cualquier momento o 
lugar podemos olvidarnos de ordenar y almacenar cientos de papeles.

\section{Motivación}
La motivación para la realización de este proyecto surge de la necesidad de 
tener un control más exhaustivo de los gastos y ahorros que a la vez nos brinde 
la comodidad de tener todo en un solo lugar y sea sencillo de usar.

------------------ COMPLETAR ------------------


\section{Objetivos} \label{sect:goals}
\subsection{Objetivo general}
Desarrollar una aplicación móvil, que usando tecnologías de reconocimiento
óptico de caracteres (OCR) y geolocalización, permita llevar a cabo un 
análisis de gastos detallado y facilite al usuario una gestión eficiente de su dinero
a través de un monedero virtual, introduciendo los datos de 
forma sencilla.

\subsection{Subobjetivos}
\begin{enumerate}
    \item Diseñar e implementar la interfaz de usuario de la aplicación facilitando
        la gestión personalizada del dinero.
    \item Integrar tecnología de reconocimiento óptico de caracteres (OCR) que 
        permita escanear y analizar tickets de compra para añadir automáticamente 
        los datos a la aplicación.
    \item ??? Desarrollar una funcionalidad que permita al usuario introducir los datos 
        de forma manual, como alternativa a la opción de escaneo.
    \item Integrar en la aplicación capacidades de geolocalización para una 
        identificación del comercio en el que se ha realizado la compra y 
         propuesta de categorización del tipo de gasto.     
    \item Implementar herramientas de visualización de datos con gráficos y resúmenes 
        automatizados que permitan analizar los patrones de gasto y ayudar al usuario 
        a realizar un seguimiento claro de los mismos.
    \item Implementar medidas de seguridad para proteger los datos personales y 
        financieros de los usuarios.
    
\end{enumerate}