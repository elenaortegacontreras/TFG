MÁS DEL ANÁLISIS DE APPS ESTO, NO??
En la actualidad las aplicaciones bancarias incorporan análisis de gastos, 
pero no son suficientes para llevar un control puesto que no se pueden 
añadir gastos que no se hayan hecho con la tarjeta de crédito. También 
incluyen planes de ahorro, pero no se pueden personalizar y adaptar a 
las necesidades de cada persona.

Además, los gastos y configuración que se hacen en la aplicación que ofrece 
el banco, no se pueden exportar a otras aplicaciones,
por lo que si se quiere cambiar de banco, se pierde toda la información, 
y con ella los planes de ahorro.

Con objetivo de solucionar la pérdida de la información al cambiar de banco, 
se propone una aplicación externa al banco. Para la unificación de gastos 
con tarjeta, pagos en efectivo, transferencias y demás operaciones, la 
aplicación propuesta permitirá añadir gastos de forma manual, y mediante 
escaneo de tickets.


\chapter{Estado del arte}
\section{Crítica al estado del arte}


\textbf{ANÁLISIS DE LAS APLICACIONES ACTUALES}
\textbf{Aplicaciones en el mercado que resuelven el problema}
Entre las aplicaciones más conocidas para la gestión de gastos se encuentran:
PLEO, REVOLUT, N26, Bnext, Fintonic, Money Pro, Spendee, Wallet, Money Manager, Money Lover, etc.

\textbf{Pleo} De pago. Prueba gratuita.

\textbf{Revolut} De pago. Prueba gratuita.

\textbf{Fintonic} Gratuita. 

\textbf{Problemas de las aplicaciones actuales}







\textbf{ELECCIÓN DE HERRAMIENTAS PARA EL DESARROLLO}
\textbf{Tecnología para el desarrollo de la aplicación}
Para desarrollar una aplicación multiplataforma, se pueden usar tecnologías como Flutter, React Native, Xamarin, etc.
https://flutter.dev/ :"Flutter es un marco de código abierto de Google para crear aplicaciones multiplataforma (móviles, 
web, de escritorio) de forma nativa a partir de una única base de código".


OCR:
necesitamos BD, reconocimiento de espaciado entre datos
por escaneo es lo más fiable, luego viene el uso de la cámara
Aclarar la imagen, lo mejor es una imagen en blanco y negro para que sea lo más sencillo

Google tiene un servicio para extraer info de un PDF no editable con DocumentAI
Hay librerías en python para usar OCR:
    - pytesseract, para reconocer caracteres 

- easyOCR
- keras-OCR
- trOCR
- docTR

\section{Propuesta}

