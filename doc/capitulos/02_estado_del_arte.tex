\chapter{Estado del arte}
\section{Crítica al estado del arte}

\section{Análisis de las posibles soluciones al problema propuesto}
En la búsqueda de soluciones al problema encontramos varias alternativas. A parte 
de la propia gestión manual, ya sea por la vía tradicional \textit{usando \textbf{papel} y boli} , 
o bien algo más automatizado? como el uso de \textbf{hojas de cálculo}, existen aplicaciones 
que nos facilitan esa tarea e incorporan resúmenes detallados. 

1. En la actualidad la mayoría de \textbf{aplicaciones bancarias} incorporan análisis de gastos. 
Aunque nos permiten ver un resumen mensual de los gastos o incluso por categorías, en 
general no se pueden personalizar demasiado. Algunos métodos de ahorro frecuentes son acumular 
de manera automática los céntimos de euro que sobran al redondear cada compra en una cuenta 
de ahorro virtual, o apartar una cantidad al final del mes. En este caso, nos centraremos 
en el ahorro guiado por la planificación, anticipación y previsión de gastos.

???????????????????
¿Cómo indico que he preguntado a clientes potenciales para la app?
?????????????????

Primero analizamos las características más interesantes (en relación al proyecto) 
de las aplicaciones de los primeros en la lista 
de bancos españoles más grandes https://es.wikipedia.org/wiki/Anexo:Bancos_de_Espa%C3%B1a :

- \textbf{CaixaBank} En la aplicación de \textbf{ImaginBank}, con el servicio MyMonz https://www.imagin.com/app/mymonz 
se recibe un informe mensual de gastos e ingresos que podemos ver por categorías. 
En cuanto a los planes de ahorro, con Mi Hucha podemos crear retos (máximo 5 huchas diferentes) 
https://www.imagin.com/ahorro/retos-ahorro 
con aportaciones periódicas mes a mes o puntuales, pero de forma global.
No aparece la opción de establecer un plan de consumo por categoría del gasto, 
con el que nos adelantemos y seamos conscientes en cada pago del cumplimiento 
de nuestros objetivos. 

- \textbf{Banco Santander}
https://www.bancosantander.es/particulares/banca-online/apps/santander
La aplicación de este banco tiene una zona de "Análisis de gastos", que nos crea 
informes similares a los de ImaginBank. 
Por otro lado, ésta sí nos permite establecer un 
presupuesto por categorías que podemos revisar eventualmente para guiarnos en el 
cumplimiento de nuestro objetivo y recibir alertas al acercarnos al límite.

- \textbf{BBVA}
Esta última parece ser la más completa en cuanto a opciones de ahorro y análisis de gastos.

Con "Mi día a día" https://www.bbva.es/personas/banca-online/control-gastos-mi-dia-a-dia.html 
obtenemos, al igual que en el resto de aplicaciones, el resumen con los movimientos de la cuenta.
Respecto a los métodos para ahorrar existe una cuenta gratuita denominada 
cuenta metas https://www.bbva.es/personas/productos/cuentas/cuenta-ahorro-metas.html#establece-tus-datos-de-contacto-y-acceso 
en la que apartar dinero a modo de hucha para conseguir hasta un máximo de 5 objetivos. 

El apartado presupuestos https://www.bbva.es/general/salud-financiera/economia-domestica/gestor-de-gastos-y-presupuestos.html 
es el lugar donde definir una cantidad máxima de gasto deseada 
al mes y por categoría, mide con códigos de colores la evolución del consumo 
respecto al presupuesto.
Encontramos también Apartados https://www.bbva.es/finanzas-vistazo/tu-guia-bbva/app/apartados-una-nueva-forma-de-ahorrar.html#:~:text=Apartados%2C%20de%20BBVA%2C%20es%20un,manera%20m%C3%A1s%20c%C3%B3moda%20y%20eficiente. 
donde el objetivo es el mismo que en presupuestos, con la diferencia de 
que es un espacio para separar visualmente tu dinero según tus necesidades,
apartando la cantidad que quieras para cada tipo de gasto.

2. Además, los gastos y configuración que se hacen en la aplicación que ofrece 
el banco, generalmente no se pueden exportar a otras aplicaciones,
por lo que si se quiere cambiar de banco, se pierde toda la información, 
y con ella los planes de ahorro.

3. Por otro lado, existen aplicaciones de terceros que permiten unificar los gastos 
independientemente del banco al que pertenezca el usuario. Entre las más conocidas 
encontramos:
---------- cosas que investigar de cada una de las aplicaciones ----------
- coste 
- cómo se introducen los datos
- necesita credenciales de acceso al banco?
- hace analisis?
- exportación de datos??
- permite hacer objetivos por categorías?

--> 1a web caracteristicas destacables: https://n26.com/es-es/blog/9-apps-para-controlar-tus-gastos 
--> 2da web: https://www.20minutos.es/tecnologia/aplicaciones/8-apps-para-gestionar-tu-economia-y-ahorrar-a-final-de-mes-4939558/ 
--> 3a web: https://www.bbva.com/es/salud-financiera/las-10-apps-para-gestionar-y-compartir-tus-gastos/ 
--> 2da web: https://www.ionos.es/digitalguide/online-marketing/vender-en-internet/las-mejores-apps-para-controlar-tus-gastos/ 

\textbf{Fintonic} https://www.fintonic.com/es-ES/inicio/ 
Un aspecto que puede causar reticencia a usar la aplicación es que necesita 
las claves de acceso a la cuenta bancaria (aunque sean de solo lectura) 
para acceder a los datos relacionados con los movimientos de la cuenta.


\textbf{Money Manager Expense & Budget} 

\textbf{Daily Budget Original} 

\textbf{Wally} 

\textbf{Wallet} 

\textbf{1Money} 

\textbf{Easy Home Finance} 

\textbf{SayMoney} 

\textbf{Moneylover}

\textbf{Money Hero}

\textbf{Spendee}

\textbf{Bluecoins}


siguiente tarea:
AHORA ANALIZAR PARA CADA OPCIÓN LAS APLICACIONES MÁS DESTACABLLES E INCONVENIENTES.


\textbf{Solución planteada / LO QUE VA A BUSCAR MI APP:}
Para guiarme en lo que quiero que tenga y por qué:
https://www.bbva.es/finanzas-vistazo/ef/ahorro/como-hacer-para-ahorrar-dinero-y-no-gastarlo.html

Ninguna de las opciones invetigadas previamente incluye todos las funcionalidades 
que se proponen como requisitos en este proyecto, por lo que se plantea la creación
de una aplicación que englobe todas ellas.

opción de establecer un plan de consumo por categoría del gasto, con el que nos adelantemos y 
seamos conscientes en cada pago del cumplimiento de nuestros objetivos.a

Proponiendo las siguientes soluciones a cada uno de los problemas descritos:

Siendo la meta mejorar la salud financiera de los usuarios, se plantea una aplicación que:

Respondiendo a cada inconveniente antes descrito se plantea una aplicación que 
agrupe las siguientes soluciones / características?:


1 --> ADAPTABILIDAD Y PERSONALIZACIÓN: 
- Lo de establecer planes de ahorro por categoría decir que lo hace
Adaptándose así a las necesidades de cada usuario.

2 -3 --> UNICIDAD / UNIFICACIÓN?
Con objetivo de solucionar la pérdida de la información al cambiar de banco, 
se propone optar' por una \textbf{aplicación externa al banco}. Para la unificación de gastos 
con tarjeta, pagos en efectivo, transferencias y demás operaciones, la 
aplicación propuesta permitirá añadir gastos de forma manual, y mediante 
escaneo de tickets. 



4 --> AUTOENGAÑO
Monedero virtual que almacena parte de los ingresos.


\textbf{ANÁLISIS DE LAS APLICACIONES ACTUALES}
\textbf{Aplicaciones en el mercado que resuelven el problema}
Entre las aplicaciones más conocidas para la gestión de gastos se encuentran:
PLEO, REVOLUT, N26, Bnext, Fintonic, Money Pro, Spendee, Wallet, Money Manager, Money Lover, etc.

\textbf{Pleo} De pago. Prueba gratuita.

\textbf{Revolut} De pago. Prueba gratuita.

\textbf{Fintonic} Gratuita. 

\textbf{Problemas de las aplicaciones actuales}







\textbf{ELECCIÓN DE HERRAMIENTAS PARA EL DESARROLLO}
\textbf{Tecnología para el desarrollo de la aplicación}
Para desarrollar una aplicación multiplataforma, se pueden usar tecnologías como Flutter, React Native, Xamarin, etc.
https://flutter.dev/ :"Flutter es un marco de código abierto de Google para crear aplicaciones multiplataforma (móviles, 
web, de escritorio) de forma nativa a partir de una única base de código".


OCR:
necesitamos BD, reconocimiento de espaciado entre datos
por escaneo es lo más fiable, luego viene el uso de la cámara
Aclarar la imagen, lo mejor es una imagen en blanco y negro para que sea lo más sencillo

Google tiene un servicio para extraer info de un PDF no editable con DocumentAI
Hay librerías en python para usar OCR:
    - pytesseract, para reconocer caracteres 

- easyOCR
- keras-OCR
- trOCR
- docTR

\section{Propuesta}

