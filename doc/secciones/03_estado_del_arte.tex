\chapter{Estado del arte}
\section{Crítica al estado del arte}

\section{Análisis de soluciones actualmente disponibles al problema planteado}
En la búsqueda de soluciones al problema encontramos varias alternativas. A parte 
de la propia gestión manual, ya sea por la vía tradicional \textit{usando \textbf{papel} y boli} , 
o bien algo más estructurado como el uso de \textbf{hojas de cálculo}, existen aplicaciones 
que nos facilitan esta tarea e incorporan resúmenes detallados. 

1. En la actualidad la mayoría de \textbf{aplicaciones bancarias} incorporan análisis de gastos. 
Aunque nos permiten ver un resumen mensual de los gastos o incluso por categorías, en 
general no se pueden personalizar demasiado. Algunos métodos de ahorro frecuentes son acumular 
de manera automática los céntimos de euro que sobran al redondear cada compra en una cuenta 
de ahorro virtual, o apartar una cantidad al final del mes. En este caso, nos centraremos 
en el ahorro guiado por la planificación, anticipación y previsión de gastos.

Primero analizamos las características más interesantes (en relación al proyecto) 
de las aplicaciones de los primeros bancos españoles más grandes según 
este\cite{este} artículo en wikipedia:

-----------
¿Cómo indico que he preguntado a clientes potenciales para la app? para 
conocer las aplicaciones que usan
-----------


\begin{itemize}

    \item \textbf{CaixaBank} 
    En la aplicación ImaginBank, con el servicio \textit{MyMonz\cite{MyMonz}}
    se recibe un informe mensual de gastos e ingresos que podemos ver por categorías. 

    En cuanto a los planes de ahorro, en el apartado \textit{Mi Hucha} podemos crear retos\cite{retos} (máximo 5 huchas diferentes)
    con aportaciones periódicas mes a mes o puntuales, pero de forma global.
    A diferencia de otras aplicaciones similares, no aparece la opción de establecer un plan de 
    consumo por categoría del gasto que nos ayude a planificarnos.

    \item \textbf{Banco Santander}
    La aplicación del Banco Santander\cite{Santander} tiene una zona de \textit{Análisis de gastos}, que nos crea 
    informes similares a los de ImaginBank. 

    Por otro lado, ésta sí nos permite establecer un 
    presupuesto por categorías que podemos revisar eventualmente para guiarnos en el 
    cumplimiento de nuestro objetivo y recibir alertas al acercarnos al límite.

    \item \textbf{BBVA}
    Esta última parece ser la más completa en cuanto a opciones de ahorro y análisis de gastos.

    Con \textit{Mi día\cite{día} a día}  
    obtenemos, al igual que en el resto de aplicaciones, el resumen con los movimientos de la cuenta.
    Respecto a los métodos para ahorrar existe una cuenta gratuita denominada 
    cuenta metas\cite{metas} 
    en la que apartar dinero a modo de hucha para conseguir hasta un máximo de 5 objetivos. 

    El apartado Presupuestos\cite{Presupuestos} 
    es el lugar donde definir una cantidad máxima de gasto deseada 
    al mes y por categoría, mide con códigos de colores la evolución del consumo 
    respecto al presupuesto.

    Encontramos también Apartados\cite{Apartados} 
    donde el objetivo es el mismo que en presupuestos, con la diferencia de 
    que es un espacio para separar visualmente tu dinero según tus necesidades,
    apartando la cantidad que quieras para cada tipo de gasto.

\end{itemize}


2- A pesar de que se están incluyendo opciones para importar los 
datos de movimientos en la cuenta, las configuraciones que se hacen dentro de la aplicación 
que ofrece el banco generalmente no se pueden exportar a otras aplicaciones.
Si se quiere cambiar de banco se pierde toda la información 
y con ella los planes de ahorro o presupuestos personalizados.

3. Por otro lado, existen \textbf{aplicaciones de terceros} que permiten unificar los gastos 
independientemente del banco al que pertenezca el usuario. Entre las más conocidas 
encontramos:

\begin{itemize}
    \item \textbf{Fintonic} https://www.fintonic.com/es-ES/inicio/ \\
    El usuario agrega las cuentas bancarias que desee para que la aplicación 
    acceda a los datos relacionados con los movimientos de la cuenta. 
    Un aspecto que puede causar cierta reticencia es que para ello Fintonic necesita
    las claves de acceso a la cuenta bancaria (aunque sean de solo lectura).
    Tiene una interfaz intuitiva, y recoge, clasifica y analiza los gastos.
    Introduce los datos de manera automática por medio de las claves proporcionadas. 
    Gratuito, con publicidad.

    \item \textbf{Money Manager}  https://moneymanagerapp.com/ \\
    Solo se encuentra disponible en inglés. Ofrece una gran cantidad de opciones
    para analizar gastos e incluye la opción de añadir fotos a las transacciones. 
    Se deben introducir los datos manualmente. 
    Dispone de una versión gratuita con funciones básicas.

    \item \textbf{Bluecoins} https://www.bluecoinsapp.com/ \\
    Plantea una estructura interesante para mayor detalle de presupuestos. 
    Se pueden configurar presupuestos por categorías y subcategorías relacionadas, por ejemplo 
    los gastos del coche, se pueden separar en gasolina, seguro, mantenimiento, etc. 
    Permite añadir gastos manualmente o importarlos desde un archivo csv. 
    Ofrece versión gratuita con anuncios.

    \item \textbf{Money Hero} https://moneyhero.site/es  \\
    Dispone de objetivos de ahorro e indica el progreso y recomendaciones 
    diarias hasta alcanzarlo. 
    En base al ritmo de consumo estima hasta cuándo dispondremos 
    de dinero suficiente, como guía para gastarlo de manera responsable. 
    El usuario debe introducir todos los ingresos y gastos a mano. 
    Dispone de versión gratuita con algunas limitaciones y anuncios.
\end{itemize}

\section{Propuesta}
Se plantea la creación de una aplicación móvil (con desarrollo multiplataforma) 
para mejorar la salud financiera de los usuarios. 

Resolviendo algunos de los problemas descritos en el apartado anterior, 
se pretende que reúna funcionalidades específicas para la 
gestión de gastos, las cuales reúnen:
\begin{itemize}
\item incorportación sencilla de gastos e ingresos
\item creación de planes de ahorro y presupuestos personalizables
\item generación de análisis y resúmenes automáticos sobre el consumo
\end{itemize}


Características:

INTEGRACIÓN DE DATOS:
Con objetivo de solucionar la pérdida de la información al cambiar de banco, 
se propone optar por una \textbf{aplicación externa al banco}. 
Para integrar los datos que se deseen ya sea con tarjeta, pagos en efectivo, transferencias 
y demás operaciones, la aplicación propuesta permitirá añadir gastos de forma manual 
y mediante escaneo de tickets. 

ACCESIBILIDAD Y FACILIDAD DE USO:
Se diseñará para ser usada por parte de usuarios de todas las edades?.

PERSONALIZACIÓN: 
Permitirá crear planes de ahorro / consumo ? por categoría del gasto, con el que nos adelantemos y 
seamos conscientes en cada pago del cumplimiento de nuestros objetivos.

AUTOMATIZACIÓN:
Generación de análisis y resúmenes automáticos sobre el consumo.





\section{Elección de herramientas para el desarrollo de la aplicación}
Para desarrollar una aplicación multiplataforma, se pueden usar tecnologías como Flutter, React Native, Xamarin, etc.
https://flutter.dev/ :"Flutter es un marco de código abierto de Google para crear aplicaciones multiplataforma (móviles, 
web, de escritorio) de forma nativa a partir de una única base de código".


OCR:
necesitamos BD, reconocimiento de espaciado entre datos
por escaneo es lo más fiable, luego viene el uso de la cámara
Aclarar la imagen, lo mejor es una imagen en blanco y negro para que sea lo más sencillo

Google tiene un servicio para extraer info de un PDF no editable con DocumentAI
Hay librerías en python para usar OCR:
    - pytesseract, para reconocer caracteres 

- easyOCR
- keras-OCR
- trOCR
- docTR

