\chapter{Conclusiones y trabajos futuros}

\section{Objetivos} \label{sect:goals}
\textit{El objetivo principal de este proyecto es desarrollar una aplicación web (\textbf{SIGMA}: Sistema Inteligente de Gestión Monetaria Automatizada) 
para la gestión financiera. Usando tecnologías de reconocimiento
óptico de caracteres (OCR) y geolocalización, permite al usuario llevar a cabo un 
análisis de gastos detallado y facilita una gestión eficiente de su dinero
a través de un monedero virtual, introduciendo los datos de 
forma sencilla.}

Para el cumplimiento de este objetivo general se plantean los siguientes objetivos específicos:
\begin{enumerate}
    \item Analizar las herramientas de gestión financiera existentes en el mercado para su aplicación en el desarrollo del proyecto.  
    \item Plantear un diseño escalable y desarrollar una arquitectura de software modular para facilitar la integración de futuras funcionalidades o mejoras. 
    \item Diseñar e implementar la interfaz de usuario para permitir la creación 
         de transacciones (ingresos, gastos y ahorros), categorías de gasto y objetivos de ahorro personalizables y modificables.
    \item Implementar herramientas de visualización de datos por medio de mapas, gráficos y resúmenes automatizados para analizar los patrones de gasto y realizar un seguimiento claro de los mismos.
    \item Estudiar e integrar tecnología de reconocimiento óptico de caracteres (OCR) y de reconocimiento de patrones en un texto para escanear tickets de compra; como punto de partida al procesamiento del texto obtenido.
    \item Implementar un sistema que permita automatizar la inserción de gastos en la aplicación, extrayendo datos relevantes de los tickets de compra para que la interverción por parte del usuario en el proceso sea mínima.
    \item Facilitar la búsqueda de comercios. Integrar en la aplicación capacidades de geolocalización para identificar en las transacciones el comercio en el que se ha realizado dicho gasto.   
    
\end{enumerate}

\section{Conclusiones}
\section{Conclusiones}
A lo largo del desarrollo de este proyecto, se han alcanzado satisfactoriamente los objetivos establecidos en la fase inicial. Estos objetivos sirvieron como guía para asegurar que cada funcionalidad cumpliera con los requerimientos del usuario y aportara una solución eficaz en el ámbito de la gestión financiera. A continuación, se detallan las conclusiones obtenidas en relación con cada uno de los objetivos planteados:

\begin{itemize}
    \item \textbf{Análisis de herramientas de gestión financiera}: El estudio de las herramientas existentes en el mercado permitió identificar las características y limitaciones más relevantes en el área de la gestión financiera personal. Gracias a este análisis, se establecieron funcionalidades clave para la aplicación, con un enfoque en simplicidad, usabilidad y flexibilidad.

    \item \textbf{Arquitectura modular y escalable}: Se ha diseñado e implementado una arquitectura de software modular, cumpliendo con el objetivo de facilitar la escalabilidad del proyecto. Esto asegura que futuras integraciones, como nuevas funcionalidades o mejoras en el rendimiento, puedan añadirse sin afectar a los componentes existentes. La arquitectura modular permite una mayor adaptabilidad a las necesidades de los usuarios a lo largo del tiempo.

    \item \textbf{Diseño de la interfaz de usuario}: La interfaz permite a los usuarios gestionar sus transacciones de ingresos, gastos y ahorros de manera eficiente. Se implementaron opciones para categorizar gastos y definir objetivos de ahorro personalizables, permitiendo a los usuarios tener un control claro sobre su situación financiera. Se cumple, por tanto, el objetivo de ofrecer una experiencia de usuario intuitiva y de bajo esfuerzo en la creación y modificación de transacciones y categorías.

    \item \textbf{Visualización de datos mediante gráficos, mapas y resúmenes}: El sistema de visualización de datos proporciona gráficos y resúmenes automáticos de gastos, ofreciendo a los usuarios un análisis visual y detallado de sus patrones de consumo. Además, la visualización de gastos en mapas ayuda a los usuarios a relacionar sus gastos con ubicaciones geográficas específicas, lo cual contribuye a una mejor comprensión y seguimiento de los hábitos de consumo.

    \item \textbf{Integración de tecnología OCR y procesamiento de tickets}: La integración de la tecnología OCR permite escanear tickets de compra y extraer toda la información del ticket de manera eficiente. Esta funcionalidad es el punto de partida hacia un procesamiento automatizado de los datos de los tickets, reduciendo la necesidad de intervención manual por parte del usuario. Este logro representa un avance en la automatización y precisión de la gestión de gastos.

    \item \textbf{Automatización en la inserción de gastos}: Se implementó un sistema de automatización que extrae los datos relevantes de los tickets de compra, lo que minimiza la intervención del usuario en el registro de sus gastos. Esto permite a los usuarios registrar sus gastos de forma rápida y cómoda, mejorando la eficiencia del proceso y cumpliendo así con el objetivo de facilitar la inserción de datos en la aplicación.

    \item \textbf{Geolocalización y búsqueda de comercios}: La integración de la funcionalidad de geolocalización permite identificar y registrar el comercio asociado a cada transacción en función de la ubicación. Esto facilita a los usuarios la identificación de sus transacciones por ubicación y les brinda información adicional a localizar sus gastos en los diferentes comercios. Esta funcionalidad contribuye a una mayor precisión y detalle en el análisis de gastos.

\end{itemize}

Por tanto, este proyecto ha logrado desarrollar una solución completa y eficiente en el ámbito de la gestión financiera personal, aportando una herramienta que integra tecnologías avanzadas y proporciona una experiencia de usuario optimizada. Esta aplicación representa una contribución al estado del arte en la gestión de finanzas personales, ya que combina funciones de análisis de datos, automatización y geolocalización, adaptadas a las necesidades de los usuarios.

Durante el desarrollo de este proyecto, he tenido la oportunidad de enriquecerme en múltiples aspectos. Crear esta solución desde cero y de manera autónoma me ha permitido mejorar significativamente mis habilidades de análisis, ya que fue necesario plantear una construcción completa en el inicio que tuviera sentido como una unidad, para poder evaluar y determinar con claridad las funcionalidades y restricciones necesarias en cada fase que se implementaría. Por otro lado, en la búsqueda de soluciones y en la implementación de las mismas, he tenido la oportunidad de aprender y aplicar nuevas tecnologías y herramientas, lo que ha ampliado mi conocimiento y experiencia en el desarrollo de aplicaciones web.

Además, fue particularmente gratificante encontrar una solución eficaz para una problemática que enfrentaba personalmente y descubrir que esta también resultaba útil para personas de mi entorno. Compartir y discutir los primeros planteamientos sobre la aplicación con otras personas me ayudó a refinar y enfocar las necesidades del proyecto, logrando una mejor comprensión de sus objetivos y una implementación más alineada con las expectativas de los usuarios.

Este proyecto representó un reto de principio a fin, que finalmente ha cumplido con las expectativas y me ha permitido adquirir una valiosa experiencia en el desarrollo de software, desde la concepción de la idea hasta la implementación de una solución funcional y eficiente. En resumen, esta experiencia me ha dejado no solo con nuevas competencias técnicas, sino también con una comprensión más amplia del proceso de desarrollo y de la utilidad de una solución bien diseñada desde el primer momento.


\section{Trabajos futuros}
...

En la actualidad, cada vez son más los comercios que ofrecen la posibilidad de recibir 
el ticket en formato digital. En un futuro probablemente todos los tickets serán 
digitales para reducir el impacto ambiental del uso del papel y abaratar costes.

Por lo que se incluirá en la aplicación la opción de escaneo de tickets digitales.


Versión gratuita no almacena tickets.
Versión de pago para amortizar el almacenaje, que sí permita almacenar todos los tickets.