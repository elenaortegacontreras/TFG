\chapter{Planificación}

La planificación en el desarrollo de un proyecto es un aspecto clave para organizar las tareas, trabajar de manera eficiente y garantizar el cumplimiento de plazos y objetivos.

\section{Metodología utilizada}
A parte de las herramientas concretas que se verán con más detalle en capítulos posteriores, para lograr una planificación adecuada se han usado las herramientas y metodologías que se describen en esta sección.

\subsection{Control de versiones}
Para el control de versiones se ha utilizado \textit{Git}, un sistema distribuido que permite llevar un registro de los cambios realizados en el código fuente de un proyecto. Es una herramienta muy potente y versátil que ha ayudado a completar las tareas de desarrollo y documentación de manera organizada y documentada. Este proyecto se encuentra alojado en \textit{GitHub}, una plataforma de desarrollo colaborativo que utiliza \textit{Git} como sistema de control de versiones; lo que facilitaría la colaboración y el trabajo en equipo si en el futuro alguien decide participar en el desarrollo del proyecto.

\subsection{Metodología de desarrollo ágil}
Las metodologías ágiles son un enfoque flexible y adaptativo para la gestión de proyectos, especialmente en desarrollo de software. Entre otros, se caracterizan por iteraciones cortas, aceptación del cambio en los requisitos durante el desarrollo y entregas continuas de producto (con el objetivo de satisfacer al cliente con software de valor desde el inicio) \cite{agileprinciples}.

\textit{Kanban} es una metodología ágil que se enfoca en la gestión visual del flujo de trabajo. Utiliza un tablero con columnas que representan diferentes etapas del proceso (como el trabajo realizado, lo que está en proceso y tareas futuras). Los elementos del proyecto se mueven a lo largo del tablero, y el desarrollador se centra en limitar las tareas en progreso para mejorar la productividad y evitar la sobrecarga \cite{majkamastering}.

Se ha optado por el uso de \textit{Kanban} por su flexibilidad y simplicidad. Permite gestionar el proyecto sin imponer plazos fijos, algo útil cuando trabajas solo y necesitas adaptar tu ritmo según las circunstancias y la carga de trabajo.

Existen herramientas que nos facilitan la implementación de esta metodología, como \textit{Trello}\footnote{\url{https://trello.com/es}} o \textit{Click Up}\footnote{\url{https://clickup.com/es-ES}}, con tableros personalizados y colaborativos para organizar tareas de proyectos.

\section{Calidad / tests (doc, back y front)/ ?}

\section{Temporización // Cronograma y planificación}

\section{Recursos y materiales}

\section{Presupuesto}

\section{Seguimiento del desarrollo}
