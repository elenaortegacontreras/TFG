\chapter{Planificación}

La planificación en el desarrollo de un proyecto es un aspecto clave para organizar las tareas, trabajar de manera eficiente y garantizar el cumplimiento de plazos y objetivos.

\section{Metodología utilizada}
Las metodologías ágiles son un enfoque flexible y adaptativo para la gestión de proyectos, especialmente en desarrollo de software. Entre otros, se caracterizan por iteraciones cortas, aceptación del cambio en los requisitos durante el desarrollo y entregas continuas de producto (con el objetivo de satisfacer al cliente con software de valor desde el inicio)\cite{agileprinciples}.

\textit{Kanban} es un enfoque ágil que destaca por la gestión visual del flujo de trabajo. Utiliza un tablero con columnas que representan diferentes etapas del proceso (como el trabajo realizado, lo que está en proceso y tareas futuras). Los elementos del proyecto se mueven a lo largo del tablero; esto facilita al desarrollador trabajar de acuerdo al enfoque de Kanban: el desarrollador se centra en limitar las tareas en progreso para mejorar la productividad y evitar la sobrecarga\cite{majkamastering}.

Se ha optado por el uso de \textit{Kanban} por su flexibilidad y simplicidad. Permite gestionar el proyecto sin imponer plazos fijos, algo útil cuando trabajas solo y necesitas adaptar tu ritmo según las circunstancias y la carga de trabajo. Además de la gestión visual, Kanban, al facilitar la priorización de tareas y la realización de entregas continuas, permite avanzar hacia objetivos bien definidos.

Existen herramientas de gestión de proyectos que nos facilitan la implementación de esta metodología, como \textit{Trello}\footnote{\url{https://trello.com/es}} o \textit{Click Up}\footnote{\url{https://clickup.com/es-ES}}, con tableros personalizados y colaborativos para organizar tareas de proyectos.


\section{Temporización}
Siguiendo el enfoque \textit{Kanban}, se ha establecido un tablero (como se puede ver en la Figura \ref{fig:tablero_kanban}) con las tareas a realizar en el proyecto. Se definen hitos de desarrollo cuyo cumplimiento supone el final de un milestone (definición de milestones en el capítulo de análisis\ref{sec:milestones}) y el inicio del siguiente; los milestones (o hitos) incluirán un desglose de tareas, que podrán incluir la implementación de funcionalidades, resolución de problemas encontrados y realización de pruebas. 

De esta forma el final del milestone se puede marcar con la finalización de todas las tareas asociadas a él, lo que supone un producto entregable funcional y de valor para el cliente. 

Los milestones guían la planificación del proyecto y permiten el seguimiento temporal del mismo, están sujetos a cambios según las necesidades del proyecto.

\begin{figure}[ht!]
    \centering
    \includegraphics[width=\linewidth]{imagenes/tablero_kanban.jpg}
    \caption{Captura de pantalla del tablero kanban en \textit{Trello}.}
    \label{fig:tablero_kanban}
\end{figure}


\section{Costes}
En este capítulo se describen los recursos y materiales usados en el desarrollo del proyecto, junto con los costos asociados a cada uno de ellos.

\subsection{Recursos humanos}

En el desarrollo de la aplicación web \textit{SIGMA}, se ha contado con un desarrollador que ha hecho también las labores de diseño para la interfaz y redacción de este documento.

\subsection{Hardware y software}
Como recursos materiales, han sido utilizados un ordenador personal portátil y un monitor (Tabla \ref{tab:costes-hardware}),
propiedad del desarrollador, los cuales han sido de gran utilidad para las tareas de diseño y desarrollo al permitir el visionado de diferentes recursos simultáneamente.

\begin{table}[H]
    \begin{center}
    \begin{tabular}{| l | c | c | c |}
        \hline
        \textbf{Hardware} & \textbf{Coste} \\ \hline
        Acer Extensa 2511 & 589\euro \\
        Monitor LG & 132,68\euro \\ \hline
        \textbf{Coste total hardware}: & \textbf{721,68\euro} \\ \hline
    \end{tabular}
    \caption{Coste de materiales hardware.}
    \label{tab:costes-hardware}
    \end{center}
\end{table} 

Para desarrollar este proyecto no se ha usado ningún software ni biblioteca de pago con el propósito de reducir costes, por lo que el software externo no incrementa el coste final. Por otro lado, el coste del hardware anteriormente mencionado, viene condicionado por la vida útil estimada de los dispositivos. En el caso de un ordenador portátil y un monitor, se estiman entre 5 y 7 años (en buenas condiciones)\cite{tecfys2023}.
\begin{equation}
    \textbf{Coste Anual} = \frac {\text{Coste Hardware}}{\text{5 años}} = \text{144,336 €/año}
\end{equation}

\begin{equation}
    \textbf{Coste Mensual} = \frac {\text{Coste Anual}}{\text{12 meses}} = \text{12,03 €/mes}
\end{equation}

\begin{equation}
    \textbf{Materiales x N Meses} = \text{Coste Mensual/mes} \times \text{8 meses} = \text{96,224 €}
\end{equation}

\subsection{Presupuesto sobre profesionales implicados}
En esta sección se detallarán los costos del proyecto. Se considerará el perfil de un puesto junior de desarrollador \textit{fullstack}, cuyo salario anual oscila entre los 17.000€ y 22.000€ \cite{glassdoor2024}.
Teniendo en cuenta que la implementación ha tomado un tiempo aproximado de 370 horas, se puede calcular un coste estimado del salario del desarrollador, el cual se detalla a continuación:

\begin{equation}
    \textbf{Salario medio fullstack junior} =  \frac {\text{19.500 €/año} }{ \text{12 meses}} = \text{1.625 €/mes}
\end{equation}

El salario por hora se calcula dividiendo el salario mensual entre 160 horas, que es el promedio de horas laborales al mes:

\begin{equation}
    \textbf{Salario por hora} = \frac {\text{1.625 €/mes}}{160 \text{ horas}} = \text{10,16 €/hora}
\end{equation}

Teniendo en cuenta las 370 horas de trabajo, se puede calcular el coste total del salario:

\begin{equation}
    \textbf{Coste total salario} = \text{370 horas} \times \text{10.16 €/hora} = \text{3.757,81 €}
\end{equation}

Para el diseño, se tendrá en cuenta el perfil de un diseñador gráfico, cuyo salario medio es de unos 18.180 € \cite{glassdoor-dis-grafico-2024}. Teniendo en cuenta que el diseño ha supuesto un aproximado de 30 horas. Partiendo de esta información, se puede calcular el coste total del salario del diseñador, el cual se detalla a continuación:

\begin{equation}
    \textbf{Salario medio diseñador gráfico} =  \frac {\text{18.180 €/año} }{ \text{12 meses}} = \text{1,515 €/mes}
\end{equation}

El salario por hora se calcula dividiendo el salario mensual entre 160 horas, que es el promedio de horas laborales al mes:

\begin{equation}
    \textbf{Salario por hora} = \frac {\text{1,515 €/mes}}{160 \text{ horas}} = \text{9,47 €/hora}
\end{equation}

Teniendo en cuenta las 30 horas de trabajo, se puede calcular el coste total del salario del diseñador:

\begin{equation}
    \textbf{Coste salario} = \text{30 horas} \times \text{9,47 €/hora} = \text{284,06 €}
\end{equation}

\subsection*{Total}

Sumando los costes asociados a los salarios y materiales usados, se obtiene el coste total aproximado del proyecto (representado en la Tabla \ref{tab:coste-total}).

\begin{table}[H]
    \begin{center}
    \begin{tabular}{| l | c | c | c |}
        \hline
        \textbf{Entidad} & \textbf{Coste} \\ \hline
        Desarrollador & 3.757,81\euro \\
        Diseñador  & 284,06\euro \\ \hline
        Hardware
        \textbf{Coste total}: & \textbf{4.041,87\euro} \\ \hline
    \end{tabular}
    \caption{Coste de materiales hardware.}
    \label{tab:coste-total}
    \end{center}
\end{table} 