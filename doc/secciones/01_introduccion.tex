\chapter{Introducción}
Este proyecto es software libre, y está liberado con la licencia \cite{gplv3}.\\

Puede encontrarse en el siguiente repositorio:
\href{https://github.com/elenaortegacontreras/TFG}{elenaortegacontreras/TFG}, 
en el cual se ha desarrollado desde el principio en abierto.

\section{Contexto. Descripción del problema}

Según su definición en la RAE \footnote{\url{https://dle.rae.es/economía}} la economía es la 
\textit{Administración eficaz y razonable de los bienes}, dicho esto y teniendo en 
cuenta que nos concierne a todos, es crucial tener control sobre ella. 
De manera individual, las decisiones financieras influyen en la calidad de vida de las personas,
un uso adecuado es esencial para evitar problemas como el endeudamiento, la 
falta de ahorro o la incapacidad de cumplir metas económicas.

En una época en la que los precios son muy elevados, pero a la vez el consumo parece estar 
desvocado, es importante administrar el dinero de manera responsable. Esto 
implica planificación y control sobre cómo y dónde lo gastamos.
Sin embargo, no es fácil hacer el seguimiento de las finanzas personales ya que se 
juntan diversos factores a tener en cuenta: podemos tener varias fuentes de ingresos y/o 
gastos, planes de ahorro, diferentes cuentas bancarias, uso de dinero tanto 
efectivo como digital, etc. por lo que en ocasiones resulta tedioso 
llevar las cuentas al día para analizar correctamente nuestro consumo.

La transición hacia lo digital está cambiando nuestra forma de interactuar con 
la información, nos brinda la posibilidad de agrupar todo en un 
único dispositivo, con el que olvidarnos de ordenar y almacenar cientos de papeles.
En el ámbito de la economía personal, la digitalización ha supuesto (entre 
otros) la creación de aplicaciones que nos ayudan a gestionar nuestros 
gastos, ahorros, inversiones, etc.

\section{Motivación}
Dado el escenario descrito, la motivación para la realización de este proyecto surge del deseo 
de contribuir a la mejora en la calidad de vida de las personas a través de un mayor control 
sobre sus decisiones financieras.

¿El párrafo que viene a continuación está bien aquí?
¿O debería ir en el apartado 3.4 Propuesta (dentro de Estado del Arte)?
El desarrollo de este Trabajo de Fin de Grado se centra en la creación de una solución digital que 
facilite el seguimiento de los gastos y el control de ahorros y presupuestos. Con este fin, 
para la introducción de gastos en la aplicación se da especial importancia a la automatización 
de procesos ?? y facilidad de uso ??. Se generan gráficos y resúmenes que permiten analizar los
patrones de gasto y reúne diferentes formas de visualización de los datos aportando información 
de valor como la categorización de los gastos, la localización de los comercios, etc.\\

------------------ ¿ COMPLETAR ?------------------


\section{Objetivos} \label{sect:goals}
\subsection{Objetivo general}
Desarrollar una aplicación móvil (\textbf{SIGMA}: Sistema Inteligente de Gestión Monetaria Automatizada) 
para la gestión financiera. Usando tecnologías de reconocimiento
óptico de caracteres (OCR) y geolocalización, permite al usuario llevar a cabo un 
análisis de gastos detallado y facilita una gestión eficiente de su dinero
a través de un monedero virtual, introduciendo los datos de 
forma sencilla.

\subsection{Subobjetivos}
\begin{enumerate}
    \item Estudio de las tecnologías actuales. Análisis de las aplicaciones 
        de gestión financiera existentes en el mercado y de las tecnologías 
        que se pueden aplicar en el desarrollo de la aplicación. Este estudio 
        permite identificar funcionalidades básicas, características que
        resulten diferenciadoras y reducción de costes en herramientas de terceros.
    \item Planteamiento de un diseño escalable. Desarrollar una arquitectura de
        software que permita añadir futuras funcionalidades o mejoras sobre lo actual 
        sin comprometer la funcionalidad o complejidad de la estructura de la aplicación
    \item Integrar tecnología de reconocimiento óptico de caracteres (OCR) que 
        permita escanear tickets de compra; como punto de partida para el procesamiento 
        del texto obtenido.
    \item Estudio de métodos de reconocimiento de patrones en un texto. Implementar 
        un sistema de reconocimiento de patrones de texto que permita identificar 
        los datos relevantes de los tickets de compra escaneados.
    \item Facilitar la búsqueda de comercios. Integrar en la aplicación capacidades de geolocalización 
        para una identificación del comercio en el que se ha realizado la compra 
        ?? y propuesta de categorización del tipo de gasto ??. Además integrar un buscador 
        independiente a la geolocalización del usuario.     
    \item Aprendizaje de herramientas en el marco de sistemas web (backend y frontend)
    \item Diseñar e implementar la interfaz de usuario de la aplicación facilitando
         la personalización en la organización del dinero. Permitir al usuario la creación 
         de categorías de gasto y objetivos de ahorro personalizados y modificables.
    \item Implementar herramientas de visualización de datos con gráficos y resúmenes 
        automatizados que permitan analizar los patrones de gasto como ayuda al usuario 
        para realizar un seguimiento claro de los mismos.
    \item ?? Implementar medidas de seguridad para proteger los datos personales y 
        financieros de los usuarios ??.
    
\end{enumerate}



?? \section{Personajes y viajes de usuario} ??
?? \section{Historias de usuario} ??
?? \section{Milestones} ??