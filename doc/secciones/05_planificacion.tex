\chapter{Planificación}

La planificación en el desarrollo de un proyecto es un aspecto clave para organizar las tareas, trabajar de manera eficiente y garantizar el cumplimiento de plazos y objetivos.

\section{Metodología utilizada}
Las metodologías ágiles son un enfoque flexible y adaptativo para la gestión de proyectos, especialmente en desarrollo de software. Entre otros, se caracterizan por iteraciones cortas, aceptación del cambio en los requisitos durante el desarrollo y entregas continuas de producto (con el objetivo de satisfacer al cliente con software de valor desde el inicio) \cite{agileprinciples}.

\textit{Kanban} es una metodología ágil que destaca por la gestión visual del flujo de trabajo. Utiliza un tablero con columnas que representan diferentes etapas del proceso (como el trabajo realizado, lo que está en proceso y tareas futuras). Los elementos del proyecto se mueven a lo largo del tablero; esto facilita al desarrollador trabajar de acuerdo al enfoque de Kanban: el desarrollador se centra en limitar las tareas en progreso para mejorar la productividad y evitar la sobrecarga \cite{majkamastering}.

Se ha optado por el uso de \textit{Kanban} por su flexibilidad y simplicidad. Permite gestionar el proyecto sin imponer plazos fijos, algo útil cuando trabajas solo y necesitas adaptar tu ritmo según las circunstancias y la carga de trabajo.

Existen herramientas de gestión de proyectos que nos facilitan la implementación de esta metodología, como \textit{Trello}\footnote{\url{https://trello.com/es}} o \textit{Click Up}\footnote{\url{https://clickup.com/es-ES}}, con tableros personalizados y colaborativos para organizar tareas de proyectos.

\section{Calidad / tests (doc, back y front)/ ?}

\section{Temporización // Cronograma y planificación}
- Asignación de prioridades a las historias de usuario y 


\section{RECURSOS Y COSTES}
En este apartado se detallan los recursos materiales empleados en el desarrollo del proyecto
y los costes asociados a los mismos.

\subsection{RECURSOS HUMANOS}

En el desarrollo de la plataforma SIGMA, se ha contado con una unica desarrolladora que ha
hecho también las labores de diseño para la interfaz, a parte de la redacción de este
documento.

\subsection{MATERIALES}

Como recursos materiales, han sido utilizados un ordenador personal portátil y un monitor,
propiedad de la autora, los cuales han sido de gran utilidad para las tareas de diseño y desarrollo al permitir el
visionado de múltiples recursos simultáneamente.

\begin{table}[H]
    \begin{tabular}{|c|c|c|ll}
    \cline{1-3}
    \multicolumn{1}{|l|}{Concepto}                                                                              & \multicolumn{1}{l|}{Coste Unitario (€)} & \multicolumn{1}{l|}{Coste Materiales (€)} &  &  \\ \cline{1-3}
    \begin{tabular}[c]{@{}c@{}}nombre portatil\end{tabular} & precio €                                   & \multirow{2}{*}{Coste total €}                  &  &  \\ \cline{1-2}
    \begin{tabular}[c]{@{}c@{}}nombre monitor\end{tabular}                            & precio€                                  &                                           &  &  \\ \cline{1-3}
    \end{tabular}
    \caption{Costes Materiales.}
    \label{tab:coste_materiales}
\end{table}

En materia de licencias, en este proyecto se ha hecho uso exclusívamente de tecnologías de código abierto con el
propósito de reducir costes. Por otro lado, el coste del hardware anteriormente mencionado, viene condicionado
por la vida útil estimada de los dispositivos. En el caso de un ordenador portátil y un monitor, se estiman entre
5 y 7 años (en buenas condiciones)\footnote{\url{https://www.tiendatr.com/blog/cada-cuanto-es-bueno-cambiar-nuestra-pantalla-del-ordenador/#:~:text=La%20durabilidad%20de%20un%20monitor,y%20cuando%20lo%20cuidemos%20bien.}}.
\footnote{\url{https://tecfys.com/blog/post/23-conoces-la-vida-media-de-tu-ordenador#:~:text=Por%20lo%20general%2C%20la%20vida,entre%20los%203%20%2D%205%20a%C3%B1os.}}.
\begin{equation}
    \textbf{Coste Anual} = \frac {\text{Coste Materiales}}{\text{5 años}} = \text{precio €/año}
\end{equation}

\begin{equation}
    \textbf{Coste Mensual} = \frac {\text{Coste Anual}}{\text{12 meses}} = \text{precio €/mes}
\end{equation}

\begin{equation}
    \textbf{Materiales x N Meses} = \text{precio mes €/mes} \times \text{N meses} = \text{precio €}
\end{equation}

------------------

\subsection{PRESUPUESTO}

En esta sección se detallará los costos del proyecto. Se considerará el perfil de un puesto junior de desarrollador fullstack,
teniendo en cuenta que el salario anual oscila entre los 17.000€ y 22.000€ \footnote{\url{https://www.glassdoor.es/Sueldos/desarrollador-full-stack-junior-sueldo-SRCH_KO0,31.htm}}.
Teniendo en cuenta que la implementación ha tomado un aproximado de 370 horas, se puede calcular un coste estimado del salario de la desarroladora, el cual se detalla a continuación:

\begin{equation}
    \textbf{Salario medio fullstack junior} =  \frac {\text{22.000 €/Año} }{ \text{12 meses}} = \text{precio mes €/Mes}
\end{equation}

El salario por hora se calcula dividiendo el salario mensual entre 160 horas, que es el promedio de horas laborales al mes:

\begin{equation}
    \textbf{Salario por Hora} = \frac {\text{precio mes €/Mes}}{160 \text{ Horas}} = \text{precio hora €/Hora}
\end{equation}

Teniendo en cuenta las 370 horas de trabajo, se puede calcular el coste total del salario:

\begin{equation}
    \textbf{Coste Salario} = \text{370 Horas} \times \text{precio hora €/Hora} = \text{salario €}
\end{equation}

Para el diseño, se tendrá en cuenta el perfil de un diseñador gráfico, cuyo salario anual oscila entre los 17.000 € - 22.000 € \footnote{\url{https://www.glassdoor.es/Sueldos/disenador-grafico-sueldo-SRCH_KO0,17.htm}}. Teniendo en cuenta que el diseño ha supuesto un aproximado de 16 horas. Partiendo de esta información, se puede calcular el coste total del salario del diseñador, el cual se detalla a continuación:

\begin{equation}
    \textbf{Salario medio Diseñador Gráfico} =  \frac {\text{19.500 €/Año} }{ \text{12 meses}} = \text{1.625 €/Mes}
\end{equation}

El salario por hora se calcula dividiendo el salario mensual entre 160 horas, que es el promedio de horas laborales al mes:

\begin{equation}
    \textbf{Salario por Hora} = \frac {\text{1.625 €/Mes}}{160 \text{ Horas}} = \text{10,16 €/Hora}
\end{equation}

Teniendo en cuenta las X horas de trabajo, se puede calcular el coste total del salario del diseñador:

\begin{equation}
    \textbf{Coste Salario} = \text{X Horas} \times \text{10,16 €/Hora} = \text{X €}
\end{equation}

Sumando los costes de salario y materiales, se obtiene el coste total del proyecto (Figura \ref{tab:coste_total}).

\begin{table}[H]
    \centering
    \begin{tabular}{|c|c|ll}
    \cline{1-2}
    \multicolumn{1}{|l|}{Concepto} & \multicolumn{1}{l|}{Coste (€)} &  &  \\ \cline{1-2}
    Salario Desarrollador           & 2.500,00                        &  &  \\ \cline{1-2}
    Salario Diseñador               & 169,60                         &  &  \\ \cline{1-2}
    Materiales                      & 113,04                       &  &  \\ \cline{1-2}
    \textbf{Total}                   & \textbf{2.782,64}               &  &  \\ \cline{1-2}
    \end{tabular}
    \caption{Coste Total del Proyecto.}
    \label{tab:coste_total}
\end{table}

