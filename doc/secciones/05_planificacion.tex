\chapter{Planificación}

La planificación en el desarrollo de un proyecto es un aspecto clave para organizar las tareas, trabajar de manera eficiente y garantizar el cumplimiento de plazos y objetivos.

\section{Metodología utilizada}
Las metodologías ágiles son un enfoque flexible y adaptativo para la gestión de proyectos, especialmente en desarrollo de software. Entre otros, se caracterizan por iteraciones cortas, aceptación del cambio en los requisitos durante el desarrollo y entregas continuas de producto (con el objetivo de satisfacer al cliente con software de valor desde el inicio) \cite{agileprinciples}.

\textit{Kanban} es una metodología ágil que destaca por la gestión visual del flujo de trabajo. Utiliza un tablero con columnas que representan diferentes etapas del proceso (como el trabajo realizado, lo que está en proceso y tareas futuras). Los elementos del proyecto se mueven a lo largo del tablero; esto facilita al desarrollador trabajar de acuerdo al enfoque de Kanban: el desarrollador se centra en limitar las tareas en progreso para mejorar la productividad y evitar la sobrecarga \cite{majkamastering}.

Se ha optado por el uso de \textit{Kanban} por su flexibilidad y simplicidad. Permite gestionar el proyecto sin imponer plazos fijos, algo útil cuando trabajas solo y necesitas adaptar tu ritmo según las circunstancias y la carga de trabajo.

Existen herramientas que nos facilitan la implementación de esta metodología, como \textit{Trello}\footnote{\url{https://trello.com/es}} o \textit{Click Up}\footnote{\url{https://clickup.com/es-ES}}, con tableros personalizados y colaborativos para organizar tareas de proyectos.

\section{Calidad / tests (doc, back y front)/ ?}

\section{Temporización // Cronograma y planificación}
- Asignación de prioridades a las historias de usuario y 

\section{Recursos y materiales}

\section{Presupuesto}

\section{Seguimiento del desarrollo}
