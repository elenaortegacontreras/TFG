\chapter{Análisis}

El desarrollo de este Trabajo de Fin de Grado se centra en la creación de una solución digital capaz de incluir todas las transacciones monetarias de un usuario, independientemente del origen de las mismas, centralizando la información en un único lugar. La aplicación facilitará el seguimiento de los gastos y el control de ahorros y presupuestos de manera sencilla para el usuario. Con este fin, para la introducción de gastos en la aplicación se da especial importancia a la automatización de procesos. Se generan gráficos y resúmenes que permiten analizar los patrones de gasto y reúne diferentes formas de visualización de los datos aportando información de valor como la categorización de los gastos o la localización de los comercios.

Una vez establecido lo que la aplicación debe lograr, se puede tomar un enfoque del trabajo guiado por las necesidades concretas de los usuarios, asegurando que las funcionalidades propuestas respondan a sus problemas reales. En esta sección se describen los personajes, historias de usuario y milestones, elementos fundamentales para orientar el diseño y desarrollo de la plataforma web. 

\section{Personajes y viajes de usuario}
Los personajes son perfiles de usuario que representan a los diferentes tipos de usuarios que interactuarán con la aplicación. Estos ayudan a guiar el diseño y el desarrollo de la aplicación teniendo en cuenta las necesidades y expectativas de los usuarios reales.

Algunos personajes que podrían beneficiarse de la aplicación descrita porque encuentran problemas con la administración de sus finanzas son:

\begin{itemize}
    \item Andrés, un estudiante universitario de primer año (fuera de su ciudad natal) ha dejado su hogar para mudarse a otra ciudad, donde estudia. Recibe mensualmente una asignación fija de dinero por parte de sus padres, la cual debe administrar cuidadosamente para cubrir sus gastos básicos (alquiler, alimentación, transporte y ocio). No tiene experiencia previa en la gestión de su propio dinero y suele gastar más en las primeras semanas del mes, quedándose con menos recursos para el resto. Quiere ser consciente de cuánto gasta día a día para poder ajustar su presupuesto si es necesario y terminar el mes sin problemas económicos.

    \item Ana es diseñadora gráfica en un departamento de marketing, tiene un empleo estable y recibe un salario mensual. Le encanta la moda y, para renovar su armario, suele comprar y vender ropa de segunda mano. Actualmente está ahorrando para comprar su primer coche, sin embargo, a veces actúa impulsivamente, adquiriendo prendas que le gustan sin considerar cómo ese gasto afectará a sus objetivos. Ana piensa que le vendría bien una herramienta para controlar sus gastos de manera más consciente; y que en el momento de duda al comprar pueda visualizar gráficos sencillos que muestren cuánto lleva gastado en el mes para mantenerse enfocada en su objetivo de ahorro para el coche.
    
    \item Daniela trabaja en una consultora de marketing. Es una persona socialmente activa que disfruta saliendo a cenar con sus amigos. Con frecuencia se ofrece a pagar la cuenta completa cuando salen en grupo (casi siempre con tarjeta); a menudo sus amigos le devuelven su parte en efectivo, lo que le genera dificultades para controlar exactamente cuánto ha gastado en estos encuentros, ya que el dinero devuelto no aparece reflejado en sus aplicaciones bancarias y le resulta complicado llevar un registro claro. Daniela desea integrar y gestionar de forma automática y precisa los pagos con tarjeta y el efectivo que recibe, para que sus finanzas reflejen con exactitud lo que gasta. Además sería útil para ella poder tener una visión clara de cuánto gasta realmente en restaurantes por medio de mapas, para evitar hacer cálculos manuales cada vez que necesita conocerlo.
    
    \item Manuel, de 57 años, es profesor de educación física en un colegio. Su trabajo no le ha forzado a indagar en el uso de las nuevas tecnologías más allá de las tareas básicas con los ordenadores de la escuela, por lo que no está familiarizado con las aplicaciones de gestión financiera digital. Aunque ha intentado usarlas, se siente abrumado por la cantidad de opciones y desconfía de introducir las claves de su banco en ellas. Manuel busca tener un lugar donde organizar sus finanzas, con una interfaz fácil de usar y sin demasiadas funciones avanzadas.
    
\end{itemize}

\section{Historias de usuario}
Las historias de usuario se enfocan en las necesidades específicas de los personajes y ayudan a definir las funcionalidades clave que se implementarán en la aplicación.

\begin{itemize}
    \item HU1: Seguimiento conjunto de pagos digitales y en efectivo\\
    Como usuario que realiza pagos en efectivo y digitales,
    quiero poder añadir mis gastos, ahorros e ingresos en un único lugar
    para tener un registro completo y preciso de mis finanzas, independientemente del método de pago.
    \item HU2: Creación de Presupuestos Personalizados\\
    Como usuario que quiere controlar su consumo,
    quiero crear presupuestos personalizados para diferentes categorías de gasto (comida, ocio, ahorro),
    para seguir un plan claro y controlar mis finanzas de manera eficiente.
    \item HU3: Creación de Objetivos de Ahorro\\
    Como usuario que busca planificar sus finanzas a largo plazo,
    quiero establecer objetivos de ahorro personalizados para diferentes metas (viajes, regalos, mejoras en el hogar),
    para motivarme a ahorrar y tener una visión clara de mis metas financieras.
    \item HU4: Análisis Automático de Consumo\\
    Como usuario que busca optimizar sus finanzas,
    quiero obtener análisis automáticos sobre mis gastos, visualizados en gráficos y categorizados,
    para entender de forma sencilla mi consumo y ajustar mi comportamiento financiero
    \item HU5: Visualización de Gastos por Localización\\
    Como usuario que quiere entender dónde gasto más,
    quiero ver mis gastos clasificados por su ubicación en un mapa,
    para tener una visión clara de los lugares donde realizo la mayor parte de mis compras y ajustar mis hábitos de consumo.
    \item HU6: Añadir gastos de forma manual o automáticamente\\ 
    Como usuario que quiere llevar un control de sus gastos y que sea cómodo insertarlos en la aplicación, quiero poder añadir gastos de forma manual y automática (escaneando tickets) para poder llevar un seguimiento de mis transacciones sin que ello implique un gran esfuerzo.
    \item HU7: Integrar funcionalidad de gestión financiera en proyecto propio\\
    Como Desarrollador que desea integrar alguna funcionalidad concreta de gestión financiera en su propia aplicación, quiero conocer las decisiones de diseño y las plataformas utilizadas en el desarrollo del proyecto \textit{Sigma}, para evitar errores de diseño en mi propia aplicación y asegurar que la integración sea coherente.
    
\end{itemize}

\section{Milestones}    
Los milestones representan momentos clave en el desarrollo del proyecto, cuando se completan funciones o características importantes. Estos hitos ayudan a asegurar el avance y pueden usarse para evaluar el progreso. Aunque los milestones pueden variar según el proyecto, en el caso de la aplicación descrita, se pueden establecer los siguientes como punto de partida:

\subsection{M0: Documentación y Diseño Inicial}
\begin{itemize}
    \item \textbf{Objetivo}: Establecer la base conceptual y técnica del proyecto.
    \item \textbf{Actividades}:
        \begin{itemize}
            \item Inicio de la documentación del proyecto y revisión del estado del arte.
            \item Desarrollo de un diseño inicial de la aplicación.
        \end{itemize}
    \item \textbf{Entregables}: Documento de especificación del proyecto, diseño preliminar y planificación de sprints futuros.
    \item \textbf{Criterios de aceptación}: Documento aprobado con requisitos definidos y diseño inicial revisado y aprobado.
\end{itemize}

\subsection{M1: Arquitectura del Backend y API Básica}
\begin{itemize}
    \item \textbf{Objetivo}: Establecer la infraestructura del backend y desarrollar la API inicial.
    \item \textbf{Actividades}:
        \begin{itemize}
            \item Estructura del backend de la aplicación.
            \item Diseño e implementación de tablas de la base de datos.
            \item Creación de los endpoints para CRUD en la base de datos (gastos).
            \item Pruebas unitarias de endpoints clave (e.g., creación de un gasto).
        \end{itemize}
    \item \textbf{Entregables}: Backend básico con endpoints funcionales para gastos, base de datos estructurada y pruebas preliminares.
    \item \textbf{Criterios de aceptación}: Todos los endpoints CRUD están operativos y se han pasado las pruebas iniciales.
\end{itemize}

\subsection{M2: Estructura del Frontend e Integración Inicial}
\begin{itemize}
    \item \textbf{Objetivo}: Desarrollar el frontend básico y conectarlo con el backend.
    \item \textbf{Actividades}:
        \begin{itemize}
            \item Creación de la estructura y componentes principales del frontend.
            \item Implementación de la interfaz de usuario y formularios para gastos.
            \item Conexión de frontend con la API y pruebas de integración.
        \end{itemize}
    \item \textbf{Entregables}: Frontend funcional conectado a la API, vistas y formularios para la gestión de gastos.
    \item \textbf{Criterios de aceptación}: Funcionalidades del frontend operativas y verificación de la integración con el backend.
\end{itemize}

\subsection{M3: Visualización de Datos}
\begin{itemize}
    \item \textbf{Objetivo}: Añadir capacidad de visualización de datos financieros (gráficos y resúmenes de gastos).
    \item \textbf{Actividades}:
        \begin{itemize}
            \item Creación de gráficos y resúmenes automáticos de gastos en el frontend.
            \item Pruebas para validar la correcta visualización y actualización de los datos.
        \end{itemize}
    \item \textbf{Entregables}: Visualización de gráficos y resumen de gastos.
    \item \textbf{Criterios de aceptación}: Gráficos y resúmenes de gastos son visibles y se actualizan automáticamente en función de los datos de la API.
\end{itemize}

\subsection{M4: Extracción Automática de Datos de Imágenes y PDFs}
\begin{itemize}
    \item \textbf{Objetivo}: Permitir la carga y análisis automático de imágenes y PDFs para extraer información de gastos.
    \item \textbf{Actividades}:
        \begin{itemize}
            \item Implementación en el backend de procesamiento de imágenes y PDFs.
            \item Funcionalidad en frontend para cargar documentos y visualizar datos extraídos.
        \end{itemize}
    \item \textbf{Entregables}: Funcionalidad completa de carga de documentos con extracción de datos automatizada.
    \item \textbf{Criterios de aceptación}: Pruebas exitosas con diferentes formatos de documentos, extracción y visualización correcta de la información.
\end{itemize}

\subsection{M5: Geolocalización de Gastos}
\begin{itemize}
    \item \textbf{Objetivo}: Visualizar la localización de los gastos y permitir búsquedas por cercanía.
    \item \textbf{Actividades}:
        \begin{itemize}
            \item Creación de tabla de municipios de España y mapeo de gastos por código postal.
            \item Implementación en backend de funcionalidades para geolocalización y búsqueda de tiendas.
            \item Visualización de gastos en un mapa en el frontend.
        \end{itemize}
    \item \textbf{Entregables}: Funcionalidad de geolocalización y visualización de gastos en mapa.
    \item \textbf{Criterios de aceptación}: Pruebas de geolocalización con localización de tiendas y representación correcta en el mapa.
\end{itemize}

\subsection{M6: Retoques de Interfaz y Funcionalidades de Ahorro}
\begin{itemize}
    \item \textbf{Objetivo}: Mejorar la usabilidad y añadir funcionalidad para gestión de ahorros.
    \item \textbf{Actividades}:
        \begin{itemize}
            \item Optimización de la interfaz de usuario con énfasis en claridad y usabilidad.
            \item Implementación de ``Monedero de Ahorro'' para control detallado de ahorros.
        \end{itemize}
    \item \textbf{Entregables}: Interfaz refinada y funcionalidad de monedero para ahorros y gastos.
    \item \textbf{Criterios de aceptación}: Verificación de la funcionalidad del monedero y aceptación de interfaz optimizada tras pruebas de usuario.
\end{itemize}

\subsection{M7: Documentación Final, Despliegue y Presentación}
\begin{itemize}
    \item \textbf{Objetivo}: Finalizar documentación, desplegar la aplicación en producción y preparar la entrega del proyecto.
    \item \textbf{Actividades}:
        \begin{itemize}
            \item Redacción de la memoria del proyecto con detalles de desarrollo y pruebas.
            \item Despliegue en servidor de producción y pruebas de rendimiento.
            \item Generación de documentación técnica y guía de usuario.
        \end{itemize}
    \item \textbf{Entregables}: Memoria del proyecto, aplicación desplegada en producción, documentación técnica y guía de usuario.
    \item \textbf{Criterios de aceptación}: Aplicación funcional en producción y documentación completa entregada y revisada.
\end{itemize}
