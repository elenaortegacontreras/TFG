\chapter{Análisis del problema}
 
\section{Propuesta}
Se plantea la creación de una aplicación móvil (con desarrollo multiplataforma) 
para mejorar la salud financiera de los usuarios. 

Se centra en el control del dinero analizando el consumo mes a mes y en el 
ahorro guiado por la planificación, anticipación y previsión de gastos. Para ello 
se introduce el concepto de un monedero virtual, que cada mes empieza de cero, se va 
llenando con los ingresos y se vacía con los gastos y los ahorros dedicados a otros objetivos.

Como solución para ciertos problemas descritos en el apartado anterior, y agrupando 
las características más interesantes de las aplicaciones analizadas, se propone la 
implementación de una plataforma que reúna funcionalidades específicas para la 
gestión de gastos, las cuales reúnen:
\begin{itemize}
\item incorporación sencilla de gastos e ingresos
\item creación de planes de ahorro y presupuestos personalizables
\item generación de análisis y resúmenes automáticos sobre el consumo 
con diferentes formas de visualización de los datos, aportando información 
de valor como la categorización de los gastos (vista de gastos por categoría) o 
la localización de los comercios (vista de gastos en el mapa).
\end{itemize}

Además, para unificar los datos de diferentes fuentes de ingresos y gastos, permite la 
distinción entre los pagos relacionados con operaciones digitales (tarjeta, transferencias, bizum) y los movimientos en efectivo.

DESVENTAJAS:
- Aplicaciones que importan datos bancarios (o app banco) no permiten editar las cantidades 
en los movimientos. ¿Cuántas veces saliendo con amigos una persona paga toda la cuenta 
y los demás le dan su parte? ¿Qué ocurre si alguien se lo paga en efectivo, y el voluntario para pagar el total, lo hizo con tarjeta? En estos casos es útil poder editar la cantidad real en un gasto, o incluso mejor, si no queremos editar el total pagado, lo ideal y más fiel a la realidad sería añadir el dinero recibido como ingreso en efectivo y el pago como gasto digital porque lo has pagado con la tarjeta. 

- Por otro lado, si el usuario prefiere solo hacer un seguimiento de sus gastos en efectivo, también le será de utilidad, puesto que está en su decisión libre de añadir solo aquello que se desee controlar, sin necesidad de incorporar en la aplicación todos los movimientos de sus  cuentas bancarias.

\section{?}
Características:

INTEGRACIÓN DE DATOS:
Con objetivo de solucionar la pérdida de la información al cambiar de banco, 
se propone optar por una \textbf{aplicación externa al banco}. 
Para integrar los datos que se deseen ya sea con tarjeta, pagos en efectivo, transferencias 
y demás tipos de operaciones, la aplicación propuesta permitirá añadir gastos de forma manual 
y mediante escaneo de tickets. 

ACCESIBILIDAD Y FACILIDAD DE USO:
Se diseñará para ser usada por usuarios de todas las edades?.

PERSONALIZACIÓN: 
Permitirá crear planes de ahorro / consumo por categoría del gasto, con el que anticiparnos y 
seamos conscientes en cada pago del cumplimiento de nuestros objetivos.

AUTOMATIZACIÓN:
Generación de análisis y resúmenes automáticos sobre el consumo.
