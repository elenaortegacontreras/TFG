%%%%%%%%%%%%%%%%%%%%%%%%%%%%%%%%%%%%%%%%%
% Short Sectioned Assignment LaTeX Template Version 1.0 (5/5/12)
% This template has been downloaded from: http://www.LaTeXTemplates.com
% Original author:  Frits Wenneker (http://www.howtotex.com)
% License: CC BY-NC-SA 3.0 (http://creativecommons.org/licenses/by-nc-sa/3.0/)
%%%%%%%%%%%%%%%%%%%%%%%%%%%%%%%%%%%%%%%%%

% \documentclass[paper=a4, fontsize=11pt]{scrartcl} % A4 paper and 11pt font size
\documentclass[11pt, a4paper]{book}
\usepackage[T1]{fontenc} % Use 8-bit encoding that has 256 glyphs
\usepackage[utf8]{inputenc}
\usepackage{fourier} % Use the Adobe Utopia font for the document - comment this line to return to the LaTeX default
\usepackage{listings} % para insertar código con formato similar al editor
\usepackage[spanish, es-tabla]{babel} % Selecciona el español para palabras introducidas automáticamente, p.ej. "septiembre" en la fecha y especifica que se use la palabra Tabla en vez de Cuadro
\usepackage{url} % ,href} %para incluir URLs e hipervínculos dentro del texto (aunque hay que instalar href)
\usepackage{graphics,graphicx, float} %para incluir imágenes y colocarlas
\usepackage[gen]{eurosym} %para incluir el símbolo del euro
\usepackage{cite} %para incluir citas del archivo <nombre>.bib
\usepackage{enumerate}
\usepackage{hyperref}
\usepackage{graphicx}
\usepackage{tabularx}
\usepackage{booktabs}
\usepackage{pdflscape}

\usepackage[table,xcdraw]{xcolor}
\hypersetup{
	colorlinks=true,	% false: boxed links; true: colored links
	linkcolor=black,	% color of internal links
	urlcolor=cyan		% color of external links
}
\renewcommand{\familydefault}{\sfdefault}
\usepackage{fancyhdr} % Custom headers and footers
\pagestyle{fancyplain} % Makes all pages in the document conform to the custom headers and footers
\fancyhead[L]{} % Empty left header
\fancyhead[C]{} % Empty center header
\fancyhead[R]{Elena Ortega Contreras} % My name
\fancyfoot[L]{} % Empty left footer
\fancyfoot[C]{} % Empty center footer
\fancyfoot[R]{\thepage} % Page numbering for right footer
%\renewcommand{\headrulewidth}{0pt} % Remove header underlines
\renewcommand{\footrulewidth}{0pt} % Remove footer underlines
\setlength{\headheight}{13.6pt} % Customize the height of the header

\usepackage{titlesec, blindtext, color}
\definecolor{gray75}{gray}{0.75}
\newcommand{\hsp}{\hspace{20pt}}
\titleformat{\chapter}[hang]{\Huge\bfseries}{\thechapter\hsp\textcolor{gray75}{|}\hsp}{0pt}{\Huge\bfseries}
\setcounter{secnumdepth}{4}
\usepackage[Lenny]{fncychap}

\definecolor{vskeywordcolor}{RGB}{86, 156, 214} % Azul para palabras clave
\definecolor{vscommentcolor}{RGB}{87, 166, 74}  % Verde para comentarios
\definecolor{vsstringcolor}{RGB}{214, 157, 133} % Rojo para cadenas de texto

\lstset{
    breaklines=true,            % Permite cortar líneas largas
    breakatwhitespace=false,    % Rompe líneas en cualquier punto, no solo en espacios
    basicstyle=\ttfamily\footnotesize,  % Define fuente de tamaño más pequeño y monoespaciada
    lineskip=-1pt,              % Reduce el espacio entre líneas
    frame=single,               % Añade un marco alrededor del código
    captionpos=b,               % Coloca el título del código abajo (opcional)
    columns=fullflexible,       % Permite ajustar el ancho de las columnas
    keepspaces=true,            % Mantiene los espacios en blanco en el código
    xleftmargin=0.5em,          % Margen izquierdo del código
    xrightmargin=0.5em,         % Margen derecho del código
    aboveskip=1em,              % Espacio vertical antes del código
    belowskip=1em,              % Espacio vertical después del código
    keywordstyle=\color{vskeywordcolor}, % Color para palabras clave
    commentstyle=\color{vscommentcolor}, % Color para comentarios
    stringstyle=\color{vsstringcolor},   % Color para cadenas de texto
    showstringspaces=false,     % No mostrar espacios en blanco en las cadenas
}

% Fuente: https://github.com/ghammock/LaTeX_Listings_JavaScript_ES6
% Definición del lenguaje JavaScript para listados
\usepackage{xcolor}
\lstdefinelanguage{JavaScript}{
  morekeywords=[1]{break, continue, delete, else, for, function, if, in,
    new, return, this, typeof, var, void, while, with},
  % Literals, primitive types, and reference types.
  morekeywords=[2]{false, null, true, boolean, number, undefined,
    Array, Boolean, Date, Math, Number, String, Object},
  % Built-ins.
  morekeywords=[3]{eval, parseInt, parseFloat, escape, unescape},
  sensitive,
  morecomment=[s]{/*}{*/},
  morecomment=[l]//,
}

\begin{document}

	% Plantilla portada UGR
	\input{portada/portada}

	% Plantilla prefacio UGR
	\thispagestyle{empty}

\begin{center}
{\large\bfseries SIGMA \\ Sistema Inteligente de Gestión Monetaria Automatizada }\\
\end{center}
\begin{center}
Elena Ortega Contreras\\
\end{center}

%\vspace{0.7cm}

\vspace{0.5cm}
\noindent\textbf{Palabras clave}: \textit{software libre, fintech, gestión financiera, Reconocimiento Optico de Caracteres, Geolocalización, diseño de interfaz de usario, análisis, enfoque ágil, cliente-servidor, web}
\vspace{0.7cm}

\noindent\textbf{Resumen}\\

En una época en la que los precios son muy elevados, pero a la vez el consumo parece estar desvocado, es importante administrar el dinero de manera responsable. Esto implica planificación y control sobre cómo y dónde se gasta.
Sin embargo, no es fácil hacer el seguimiento de las finanzas personales ya que se 
juntan diversos factores a tener en cuenta: tener varias fuentes de ingresos y/o 
gastos, planes de ahorro, diferentes cuentas bancarias, uso de dinero tanto 
efectivo como digital, etc. por lo que en ocasiones resulta tedioso 
llevar las cuentas al día para analizar correctamente el consumo.

Ante la diversidad de métodos de pago mencionada y la necesidad de una gestión eficiente de las finanzas, este proyecto propone unificar la información de todas las fuentes de ingresos y gastos del usuario. Al ofrecer una visión global y realista de su situación económica, la aplicación permitirá a los usuarios tener un control centralizado sobre sus finanzas, facilitando la toma de decisiones informadas y mejorando su capacidad para planificar, anticiparse y gestionar sus recursos financieros de manera más efectiva.

Se centra en el control del dinero analizando el consumo mes a mes y en el ahorro guiado por la planificación, anticipación y previsión de gastos. Para ello se introduce el concepto de un monedero virtual, que cada mes empieza de cero, se va llenando con los ingresos y se vacía con los gastos y los ahorros dedicados a determinados objetivos.	

\cleardoublepage

\begin{center}
	{\large\bfseries Same, but in English}\\
\end{center}
\begin{center}
	Elena Ortega Contreras\\
\end{center}
\vspace{0.5cm}
\noindent\textbf{Keywords}: \textit{open source}, \textit{floss}
\vspace{0.7cm}

\noindent\textbf{Abstract}\\


\cleardoublepage

\thispagestyle{empty}

\noindent\rule[-1ex]{\textwidth}{2pt}\\[4.5ex]

D. \textbf{José Manuel Soto Hidalgo}, Profesor del departamento de Ingeniería de Computadores, Automática y Robótica

\vspace{0.5cm}

\textbf{Informo:}

\vspace{0.5cm}

Que el presente trabajo, titulado \textit{\textbf{SIGMA: Sistema Inteligente de Gestión Monetaria Automatizada}},
ha sido realizado bajo mi supervisión por \textbf{Elena Ortega Contreras}, y autorizo la defensa de dicho trabajo ante el tribunal
que corresponda.

\vspace{0.5cm}

Y para que conste, expiden y firman el presente informe en Granada a Noviembre de 2024.

\vspace{1cm}

\textbf{El/la director(a)/es: }

\vspace{5cm}

\noindent \textbf{(José Manuel Soto Hidalgo)}

\chapter*{Agradecimientos}

Poner aquí agradecimientos...



	% Índice de contenidos
	\newpage
	\tableofcontents

	% Índice de imágenes y tablas
	\newpage
	\listoffigures

	% Si hay suficientes se incluirá dicho índice
	\listoftables 
	\newpage

	% Introducción 
	\chapter{Introducción}
Este proyecto es software libre, y está liberado con la licencia \cite{gplv3}.\\

Puede encontrarse en el siguiente repositorio:
\href{https://github.com/elenaortegacontreras/TFG}{elenaortegacontreras/TFG}, 
en el cual se ha desarrollado desde el principio en abierto.

\section{Contexto. Descripción del problema}

Según su definición en la RAE \footnote{\url{https://dle.rae.es/economía}} la economía es la 
\textit{Administración eficaz y razonable de los bienes}, dicho esto y teniendo en 
cuenta que nos concierne a todos, es crucial tener control sobre ella. 
De manera individual, las decisiones financieras influyen en la calidad de vida de las personas,
un uso adecuado es esencial para evitar problemas como el endeudamiento, la 
falta de ahorro o la incapacidad de cumplir metas económicas.

En una época en la que los precios son muy elevados, pero a la vez el consumo parece estar 
desvocado, es importante administrar el dinero de manera responsable. Esto 
implica planificación y control sobre cómo y dónde lo gastamos.
Sin embargo, no es fácil hacer el seguimiento de las finanzas personales ya que se 
juntan diversos factores a tener en cuenta: podemos tener varias fuentes de ingresos y/o 
gastos, planes de ahorro, diferentes cuentas bancarias, uso de dinero tanto 
efectivo como digital, etc. por lo que en ocasiones resulta tedioso 
llevar las cuentas al día para analizar correctamente nuestro consumo.

La transición hacia lo digital está cambiando nuestra forma de interactuar con 
la información, nos brinda la posibilidad de agrupar todo en un 
único dispositivo, con el que olvidarnos de ordenar y almacenar cientos de papeles.
En el ámbito de la economía personal, la digitalización ha supuesto (entre 
otros) la creación de aplicaciones que nos ayudan a gestionar nuestros 
gastos, ahorros, inversiones, etc.

\section{Motivación}
Dado el escenario descrito, la motivación para la realización de este proyecto surge del deseo 
de contribuir a la mejora en la calidad de vida de las personas a través de un mayor control 
sobre sus decisiones financieras.

¿El párrafo que viene a continuación está bien aquí?
¿O debería ir en el apartado 3.4 Propuesta (dentro de Estado del Arte)?
El desarrollo de este Trabajo de Fin de Grado se centra en la creación de una solución digital que 
facilite el seguimiento de los gastos y el control de ahorros y presupuestos. Con este fin, 
para la introducción de gastos en la aplicación se da especial importancia a la automatización 
de procesos ?? y facilidad de uso ??. Se generan gráficos y resúmenes que permiten analizar los
patrones de gasto y reúne diferentes formas de visualización de los datos aportando información 
de valor como la categorización de los gastos, la localización de los comercios, etc.\\

------------------ ¿ COMPLETAR ?------------------


\section{Objetivos} \label{sect:goals}
\subsection{Objetivo general}
Desarrollar una aplicación móvil (\textbf{SIGMA}: Sistema Inteligente de Gestión Monetaria Automatizada) 
para la gestión financiera. Usando tecnologías de reconocimiento
óptico de caracteres (OCR) y geolocalización, permite al usuario llevar a cabo un 
análisis de gastos detallado y facilita una gestión eficiente de su dinero
a través de un monedero virtual, introduciendo los datos de 
forma sencilla.

\subsection{Subobjetivos}
\begin{enumerate}
    \item Estudio de las tecnologías actuales. Análisis de las aplicaciones 
        de gestión financiera existentes en el mercado y de las tecnologías 
        que se pueden aplicar en el desarrollo de la aplicación. Este estudio 
        permite identificar funcionalidades básicas, características que
        resulten diferenciadoras y reducción de costes en herramientas de terceros.
    \item Planteamiento de un diseño escalable. Desarrollar una arquitectura de
        software que permita añadir futuras funcionalidades o mejoras sobre lo actual 
        sin comprometer la funcionalidad o complejidad de la estructura de la aplicación
    \item Integrar tecnología de reconocimiento óptico de caracteres (OCR) que 
        permita escanear tickets de compra; como punto de partida para el procesamiento 
        del texto obtenido.
    \item Estudio de métodos de reconocimiento de patrones en un texto. Implementar 
        un sistema de reconocimiento de patrones de texto que permita identificar 
        los datos relevantes de los tickets de compra escaneados.
    \item Facilitar la búsqueda de comercios. Integrar en la aplicación capacidades de geolocalización 
        para una identificación del comercio en el que se ha realizado la compra 
        ?? y propuesta de categorización del tipo de gasto ??. Además integrar un buscador 
        independiente a la geolocalización del usuario.     
    \item Aprendizaje de herramientas en el marco de sistemas web (backend y frontend)
    \item Diseñar e implementar la interfaz de usuario de la aplicación facilitando
         la personalización en la organización del dinero. Permitir al usuario la creación 
         de categorías de gasto y objetivos de ahorro personalizados y modificables.
    \item Implementar herramientas de visualización de datos con gráficos y resúmenes 
        automatizados que permitan analizar los patrones de gasto como ayuda al usuario 
        para realizar un seguimiento claro de los mismos.
    \item ?? Implementar medidas de seguridad para proteger los datos personales y 
        financieros de los usuarios ??.
    
\end{enumerate}



?? \section{Personajes y viajes de usuario} ??
?? \section{Historias de usuario} ??
?? \section{Milestones} ??

	% Estado del arte
	% 	1. Crítica al estado del arte
	% 	2. Propuesta
	\chapter{Estado del arte}
\section{Crítica al estado del arte}

\section{Análisis de las posibles soluciones al problema propuesto}
En la búsqueda de soluciones al problema encontramos varias alternativas. A parte 
de la propia gestión manual, ya sea por la vía tradicional \textit{usando \textbf{papel} y boli} , 
o bien algo más automatizado? como el uso de \textbf{hojas de cálculo}, existen aplicaciones 
que nos facilitan esa tarea e incorporan resúmenes detallados. 

1. En la actualidad la mayoría de \textbf{aplicaciones bancarias} incorporan análisis de gastos. 
Aunque nos permiten ver un resumen mensual de los gastos o incluso por categorías, en 
general no se pueden personalizar demasiado. Algunos métodos de ahorro frecuentes son acumular 
de manera automática los céntimos de euro que sobran al redondear cada compra en una cuenta 
de ahorro virtual, o apartar una cantidad al final del mes. En este caso, nos centraremos 
en el ahorro guiado por la planificación, anticipación y previsión de gastos.

???????????????????
¿Cómo indico que he preguntado a clientes potenciales para la app?
?????????????????

Primero analizamos las características más interesantes (en relación al proyecto) 
de las aplicaciones de los primeros en la lista 
de bancos españoles más grandes https://es.wikipedia.org/wiki/Anexo:Bancos_de_Espa%C3%B1a :

- \textbf{CaixaBank} En la aplicación de \textbf{ImaginBank}, con el servicio MyMonz https://www.imagin.com/app/mymonz 
se recibe un informe mensual de gastos e ingresos que podemos ver por categorías. 
En cuanto a los planes de ahorro, con Mi Hucha podemos crear retos (máximo 5 huchas diferentes) 
https://www.imagin.com/ahorro/retos-ahorro 
con aportaciones periódicas mes a mes o puntuales, pero de forma global.
No aparece la opción de establecer un plan de consumo por categoría del gasto, 
con el que nos adelantemos y seamos conscientes en cada pago del cumplimiento 
de nuestros objetivos. 

- \textbf{Banco Santander}
https://www.bancosantander.es/particulares/banca-online/apps/santander
La aplicación de este banco tiene una zona de "Análisis de gastos", que nos crea 
informes similares a los de ImaginBank. 
Por otro lado, ésta sí nos permite establecer un 
presupuesto por categorías que podemos revisar eventualmente para guiarnos en el 
cumplimiento de nuestro objetivo y recibir alertas al acercarnos al límite.

- \textbf{BBVA}
Esta última parece ser la más completa en cuanto a opciones de ahorro y análisis de gastos.

Con "Mi día a día" https://www.bbva.es/personas/banca-online/control-gastos-mi-dia-a-dia.html 
obtenemos, al igual que en el resto de aplicaciones, el resumen con los movimientos de la cuenta.
Respecto a los métodos para ahorrar existe una cuenta gratuita denominada 
cuenta metas https://www.bbva.es/personas/productos/cuentas/cuenta-ahorro-metas.html#establece-tus-datos-de-contacto-y-acceso 
en la que apartar dinero a modo de hucha para conseguir hasta un máximo de 5 objetivos. 

El apartado presupuestos https://www.bbva.es/general/salud-financiera/economia-domestica/gestor-de-gastos-y-presupuestos.html 
es el lugar donde definir una cantidad máxima de gasto deseada 
al mes y por categoría, mide con códigos de colores la evolución del consumo 
respecto al presupuesto.
Encontramos también Apartados https://www.bbva.es/finanzas-vistazo/tu-guia-bbva/app/apartados-una-nueva-forma-de-ahorrar.html#:~:text=Apartados%2C%20de%20BBVA%2C%20es%20un,manera%20m%C3%A1s%20c%C3%B3moda%20y%20eficiente. 
donde el objetivo es el mismo que en presupuestos, con la diferencia de 
que es un espacio para separar visualmente tu dinero según tus necesidades,
apartando la cantidad que quieras para cada tipo de gasto.

2. Además, los gastos y configuración que se hacen en la aplicación que ofrece 
el banco, generalmente no se pueden exportar a otras aplicaciones,
por lo que si se quiere cambiar de banco, se pierde toda la información, 
y con ella los planes de ahorro.

3. Por otro lado, existen aplicaciones de terceros que permiten unificar los gastos 
independientemente del banco al que pertenezca el usuario. Entre las más conocidas 
encontramos:
---------- cosas que investigar de cada una de las aplicaciones ----------
- coste 
- cómo se introducen los datos
- necesita credenciales de acceso al banco?
- hace analisis?
- exportación de datos??
- permite hacer objetivos por categorías?

--> 1a web caracteristicas destacables: https://n26.com/es-es/blog/9-apps-para-controlar-tus-gastos 
--> 2da web: https://www.20minutos.es/tecnologia/aplicaciones/8-apps-para-gestionar-tu-economia-y-ahorrar-a-final-de-mes-4939558/ 
--> 3a web: https://www.bbva.com/es/salud-financiera/las-10-apps-para-gestionar-y-compartir-tus-gastos/ 
--> 2da web: https://www.ionos.es/digitalguide/online-marketing/vender-en-internet/las-mejores-apps-para-controlar-tus-gastos/ 

\textbf{Fintonic} https://www.fintonic.com/es-ES/inicio/ 
Un aspecto que puede causar reticencia a usar la aplicación es que necesita 
las claves de acceso a la cuenta bancaria (aunque sean de solo lectura) 
para acceder a los datos relacionados con los movimientos de la cuenta.
Tiene una interfaz intuitiva, y recoge, clasifica y analiza los gastos.


\textbf{Money Manager}  https://moneymanagerapp.com/
Se deben introducir los datos manualmente, aunque también se pueden importar
Solo se encuentra disponible en inglés. Ofrece una gran cantidad de opciones
para analizar gastos e incluye la opción de añadir fotos a las transacciones.

\textbf{Moneyfy} 

\textbf{Wallet} 

\textbf{1Money} 
Muy completa, pero solo disponible para android.


\textbf{Money Hero} https://moneyhero.site/es  


siguiente tarea:
AHORA ANALIZAR PARA CADA OPCIÓN LAS APLICACIONES MÁS DESTACABLLES E INCONVENIENTES.


\textbf{Solución planteada / LO QUE VA A BUSCAR MI APP:}
Para guiarme en lo que quiero que tenga y por qué:
https://www.bbva.es/finanzas-vistazo/ef/ahorro/como-hacer-para-ahorrar-dinero-y-no-gastarlo.html

Ninguna de las opciones invetigadas previamente incluye todos las funcionalidades 
que se proponen como requisitos en este proyecto, por lo que se plantea la creación
de una aplicación que englobe todas ellas.

opción de establecer un plan de consumo por categoría del gasto, con el que nos adelantemos y 
seamos conscientes en cada pago del cumplimiento de nuestros objetivos.a

Proponiendo las siguientes soluciones a cada uno de los problemas descritos:

Siendo la meta mejorar la salud financiera de los usuarios, se plantea una aplicación que:

Respondiendo a cada inconveniente antes descrito se plantea una aplicación que 
agrupe las siguientes soluciones / características?:


1 --> ADAPTABILIDAD Y PERSONALIZACIÓN: 
- Lo de establecer planes de ahorro por categoría decir que lo hace
Adaptándose así a las necesidades de cada usuario.

2 -3 --> UNICIDAD / UNIFICACIÓN?
Con objetivo de solucionar la pérdida de la información al cambiar de banco, 
se propone optar' por una \textbf{aplicación externa al banco}. Para la unificación de gastos 
con tarjeta, pagos en efectivo, transferencias y demás operaciones, la 
aplicación propuesta permitirá añadir gastos de forma manual, y mediante 
escaneo de tickets. 



4 --> AUTOENGAÑO
Monedero virtual que almacena parte de los ingresos.


\textbf{ANÁLISIS DE LAS APLICACIONES ACTUALES}
\textbf{Aplicaciones en el mercado que resuelven el problema}
Entre las aplicaciones más conocidas para la gestión de gastos se encuentran:
PLEO, REVOLUT, N26, Bnext, Fintonic, Money Pro, Spendee, Wallet, Money Manager, Money Lover, etc.

\textbf{Pleo} De pago. Prueba gratuita.

\textbf{Revolut} De pago. Prueba gratuita.

\textbf{Fintonic} Gratuita. 

\textbf{Problemas de las aplicaciones actuales}







\textbf{ELECCIÓN DE HERRAMIENTAS PARA EL DESARROLLO}
\textbf{Tecnología para el desarrollo de la aplicación}
Para desarrollar una aplicación multiplataforma, se pueden usar tecnologías como Flutter, React Native, Xamarin, etc.
https://flutter.dev/ :"Flutter es un marco de código abierto de Google para crear aplicaciones multiplataforma (móviles, 
web, de escritorio) de forma nativa a partir de una única base de código".


OCR:
necesitamos BD, reconocimiento de espaciado entre datos
por escaneo es lo más fiable, luego viene el uso de la cámara
Aclarar la imagen, lo mejor es una imagen en blanco y negro para que sea lo más sencillo

Google tiene un servicio para extraer info de un PDF no editable con DocumentAI
Hay librerías en python para usar OCR:
    - pytesseract, para reconocer caracteres 

- easyOCR
- keras-OCR
- trOCR
- docTR

\section{Propuesta}



	% Especificación de requisitos
	% \input{secciones/03_requisitos.tex}

	% Análisis del problema
	% 1. Análisis de requisitos
	% 2. Análisis de las soluciones
	% 3. Solucion propuesta
	% 4. Análisis de seguridad
	\chapter{Análisis}

El desarrollo de este Trabajo de Fin de Grado se centra en la creación de una solución digital capaz de incluir todas las transacciones monetarias de un usuario, independientemente del origen de las mismas, centralizando la información en un único lugar. La aplicación facilitará el seguimiento de los gastos y el control de ahorros y presupuestos de manera sencilla para el usuario. Con este fin, para la introducción de gastos en la aplicación se da especial importancia a la automatización de procesos. Se generan gráficos y resúmenes que permiten analizar los patrones de gasto y reúne diferentes formas de visualización de los datos aportando información de valor como la categorización de los gastos o la localización de los comercios.

Una vez establecido lo que la aplicación debe lograr, se puede tomar un enfoque del trabajo guiado por las necesidades concretas de los usuarios, asegurando que las funcionalidades propuestas respondan a sus problemas reales. En esta sección se describen los personajes, historias de usuario y milestones, elementos fundamentales para orientar el diseño y desarrollo de la plataforma web. 

\section{Personajes y viajes de usuario}
Los personajes son perfiles de usuario que representan a los diferentes tipos de usuarios que interactuarán con la aplicación. Estos ayudan a guiar el diseño y el desarrollo de la aplicación teniendo en cuenta las necesidades y expectativas de los usuarios reales.

Algunos personajes que podrían beneficiarse de la aplicación descrita porque encuentran problemas con la administración de sus finanzas son:

\begin{itemize}
    \item Andrés, un estudiante universitario de primer año (fuera de su ciudad natal) ha dejado su hogar para mudarse a otra ciudad, donde estudia. Recibe mensualmente una asignación fija de dinero por parte de sus padres, la cual debe administrar cuidadosamente para cubrir sus gastos básicos (alquiler, alimentación, transporte y ocio). No tiene experiencia previa en la gestión de su propio dinero y suele gastar más en las primeras semanas del mes, quedándose con menos recursos para el resto. Quiere ser consciente de cuánto gasta día a día para poder ajustar su presupuesto si es necesario y terminar el mes sin problemas económicos. \\

    \item Ana es diseñadora gráfica en un departamento de marketing, tiene un empleo estable y recibe un salario mensual. Le encanta la moda y, para renovar su armario, suele comprar y vender ropa de segunda mano. Actualmente está ahorrando para comprar su primer coche, sin embargo, a veces actúa impulsivamente, adquiriendo prendas que le gustan sin considerar cómo ese gasto afectará a sus objetivos. Ana piensa que le vendría bien una herramienta para controlar sus gastos de manera más consciente; y que en el momento de duda al comprar pueda visualizar gráficos sencillos que muestren cuánto lleva gastado en el mes para mantenerse enfocada en su objetivo de ahorro para el coche. \\
    
    \item Daniela trabaja en una consultora de marketing. Es una persona socialmente activa que disfruta saliendo a cenar con sus amigos. Con frecuencia se ofrece a pagar la cuenta completa cuando salen en grupo (casi siempre con tarjeta); a menudo sus amigos le devuelven su parte en efectivo, lo que le genera dificultades para controlar exactamente cuánto ha gastado en estos encuentros, ya que el dinero devuelto no aparece reflejado en sus aplicaciones bancarias y le resulta complicado llevar un registro claro. Daniela desea integrar y gestionar de forma automática y precisa los pagos con tarjeta y el efectivo que recibe, para que sus finanzas reflejen con exactitud lo que gasta. Además sería útil para ella poder tener una visión clara de cuánto gasta realmente en restaurantes por medio de mapas, para evitar hacer cálculos manuales cada vez que necesita conocerlo. \\
    
    \item Manuel, de 57 años, es profesor de educación física en un colegio. Su trabajo no le ha forzado a indagar en el uso de las nuevas tecnologías más allá de las tareas básicas con los ordenadores de la escuela, por lo que no está familiarizado con las aplicaciones de gestión financiera digital. Aunque ha intentado usarlas, se siente abrumado por la cantidad de opciones y desconfía de introducir las claves de su banco en ellas. Manuel busca tener un lugar donde organizar sus finanzas, con una interfaz fácil de usar y sin demasiadas funciones avanzadas.
    
\end{itemize}

\section{Historias de usuario}
Las historias de usuario se enfocan en las necesidades específicas de los personajes y ayudan a definir las funcionalidades clave que se implementarán en la aplicación.

\begin{itemize}
    \item HU1: Seguimiento conjunto de pagos digitales y en efectivo\\
    Como usuario que realiza pagos en efectivo y digitales,
    quiero poder añadir mis gastos, ahorros e ingresos en un único lugar
    para tener un registro completo y preciso de mis finanzas, independientemente del método de pago.
    \item HU2: Creación de Presupuestos Personalizados\\
    Como usuario que quiere controlar su consumo,
    quiero crear presupuestos personalizados para diferentes categorías de gasto (comida, ocio, ahorro),
    para seguir un plan claro y controlar mis finanzas de manera eficiente.
    \item HU3: Creación de Objetivos de Ahorro\\
    Como usuario que busca planificar sus finanzas a largo plazo,
    quiero establecer objetivos de ahorro personalizados para diferentes metas (viajes, regalos, mejoras en el hogar),
    para motivarme a ahorrar y tener una visión clara de mis metas financieras.
    \item HU4: Análisis Automático de Consumo\\
    Como usuario que busca optimizar sus finanzas,
    quiero obtener análisis automáticos sobre mis gastos, visualizados en gráficos y categorizados,
    para entender de forma sencilla mi consumo y ajustar mi comportamiento financiero
    \item HU5: Visualización de Gastos por Localización\\
    Como usuario que quiere entender dónde gasto más,
    quiero ver mis gastos clasificados por su ubicación en un mapa,
    para tener una visión clara de los lugares donde realizo la mayor parte de mis compras y ajustar mis hábitos de consumo.
    \item HU6: Añadir gastos de forma manual o automáticamente\\ 
    Como usuario que quiere llevar un control de sus gastos y que sea cómodo insertarlos en la aplicación, quiero poder añadir gastos de forma manual y automática (escaneando tickets) para poder llevar un seguimiento de mis transacciones sin implicar un gran esfuerzo.

    ??? Criterios de aceptación ???:
    - La aplicación debe permitir al usuario introducir manualmente los detalles de un gasto o ingreso, incluyendo la cantidad, la fecha y una descripción.
    - El usuario debe poder diferenciar entre pagos digitales y en efectivo al registrar un gasto o ingreso.
    - La aplicación debe validar los datos introducidos por el usuario y mostrar mensajes de error en caso de que haya campos obligatorios vacíos o datos incorrectos.
    - Los gastos e ingresos añadidos manualmente deben ser almacenados en la base de datos de la aplicación para su posterior análisis y visualización.

    ??? Notas adicionales ???:
    - Aunque la automatización de la introducción de gastos e ingresos es útil, el usuario necesita la opción de añadirlos manualmente para tener un control preciso de sus finanzas.
    - La aplicación debe proporcionar una interfaz intuitiva y fácil de usar para facilitar la introducción manual de gastos e ingresos. 

    \item HU7: Integrar funcionalidad de gestión financiera en proyecto propio\\
    Como Desarrollador que desea integrar alguna funcionalidad concreta de gestión financiera en su propia aplicación, quero conocer las decisiones de diseño y las plataformas utilizadas en el desarrollo del proyecto \textit{Sigma}, para vitar errores de diseño en mi propia aplicación y asegurar que la integración sea coherente.
    
\end{itemize}

\section{Milestones}    
Los milestones representan momentos clave en el desarrollo del proyecto, cuando se completan funciones o características importantes. Estos hitos ayudan a asegurar el avance y pueden usarse para evaluar el progreso. Aunque los milestones pueden variar según el proyecto, en el caso de la aplicación descrita, se pueden establecer los siguientes como punto de partida:

\begin{itemize}
    \item M0: Repositorio y documentación inicial
    Crear el repositorio del proyecto y agregar la documentación previa a la implementación de la aplicación.

    \item M1: Arquitectura y diseño base de la aplicación
    Definir e implementar la estructura de la aplicación, incluyendo la arquitectura general y la organización del código.

    \item M2: Registro de transacciones (manual y automática)
    Implementar la funcionalidad para registrar ingresos y gastos de manera manual y automática, permitiendo diferenciar entre efectivo y transacciones digitales.

    \item M3: Presupuestos personalizables
    Desarrollar la funcionalidad que permite a los usuarios crear y gestionar presupuestos adaptados a diferentes categorías de gasto.

    \item M4: Análisis y visualización de datos
    Implementar gráficos y resúmenes automáticos que muestren los patrones de consumo, wue además permita categorizar gastos y separando por objetivos de ahorro.

    \item M5: Funcionalidad de edición de movimientos
    Añadir la capacidad para que los usuarios puedan editar y ajustar sus transacciones, permitiendo un control detallado de pagos y reembolsos.

    \item M7: Entrega de memoria. Documentación final
    Finalizar y entregar la memoria del proyecto, junto con la documentación técnica detallada para su presentación.

\end{itemize}

\section{Diagramas}



	\chapter{Planificación}

La planificación en el desarrollo de un proyecto es un aspecto clave para organizar las tareas, trabajar de manera eficiente y garantizar el cumplimiento de plazos y objetivos.

\section{Metodología utilizada}
Las metodologías ágiles son un enfoque flexible y adaptativo para la gestión de proyectos, especialmente en desarrollo de software. Entre otros, se caracterizan por iteraciones cortas, aceptación del cambio en los requisitos durante el desarrollo y entregas continuas de producto (con el objetivo de satisfacer al cliente con software de valor desde el inicio) \cite{agileprinciples}.

\textit{Kanban} es una metodología ágil que destaca por la gestión visual del flujo de trabajo. Utiliza un tablero con columnas que representan diferentes etapas del proceso (como el trabajo realizado, lo que está en proceso y tareas futuras). Los elementos del proyecto se mueven a lo largo del tablero; esto facilita al desarrollador trabajar de acuerdo al enfoque de Kanban: el desarrollador se centra en limitar las tareas en progreso para mejorar la productividad y evitar la sobrecarga \cite{majkamastering}.

Se ha optado por el uso de \textit{Kanban} por su flexibilidad y simplicidad. Permite gestionar el proyecto sin imponer plazos fijos, algo útil cuando trabajas solo y necesitas adaptar tu ritmo según las circunstancias y la carga de trabajo.

Existen herramientas de gestión de proyectos que nos facilitan la implementación de esta metodología, como \textit{Trello}\footnote{\url{https://trello.com/es}} o \textit{Click Up}\footnote{\url{https://clickup.com/es-ES}}, con tableros personalizados y colaborativos para organizar tareas de proyectos.

\section{Calidad / tests (doc, back y front)/ ?}

\section{Temporización // Cronograma y planificación}
- Asignación de prioridades a las historias de usuario y 


\section{RECURSOS Y COSTES}
En este apartado se detallan los recursos materiales empleados en el desarrollo del proyecto
y los costes asociados a los mismos.

\subsection{RECURSOS HUMANOS}

En el desarrollo de la plataforma SIGMA, se ha contado con una unica desarrolladora que ha
hecho también las labores de diseño para la interfaz, a parte de la redacción de este
documento.

\subsection{MATERIALES}

Como recursos materiales, han sido utilizados un ordenador personal portátil y un monitor,
propiedad de la autora, los cuales han sido de gran utilidad para las tareas de diseño y desarrollo al permitir el
visionado de múltiples recursos simultáneamente.

\begin{table}[H]
    \begin{tabular}{|c|c|c|ll}
    \cline{1-3}
    \multicolumn{1}{|l|}{Concepto}                                                                              & \multicolumn{1}{l|}{Coste Unitario (€)} & \multicolumn{1}{l|}{Coste Materiales (€)} &  &  \\ \cline{1-3}
    \begin{tabular}[c]{@{}c@{}}nombre portatil\end{tabular} & precio €                                   & \multirow{2}{*}{Coste total €}                  &  &  \\ \cline{1-2}
    \begin{tabular}[c]{@{}c@{}}nombre monitor\end{tabular}                            & precio€                                  &                                           &  &  \\ \cline{1-3}
    \end{tabular}
    \caption{Costes Materiales.}
    \label{tab:coste_materiales}
\end{table}

En materia de licencias, en este proyecto se ha hecho uso exclusívamente de tecnologías de código abierto con el
propósito de reducir costes. Por otro lado, el coste del hardware anteriormente mencionado, viene condicionado
por la vida útil estimada de los dispositivos. En el caso de un ordenador portátil y un monitor, se estiman entre
5 y 7 años (en buenas condiciones)\footnote{\url{https://www.tiendatr.com/blog/cada-cuanto-es-bueno-cambiar-nuestra-pantalla-del-ordenador/#:~:text=La%20durabilidad%20de%20un%20monitor,y%20cuando%20lo%20cuidemos%20bien.}}.
\footnote{\url{https://tecfys.com/blog/post/23-conoces-la-vida-media-de-tu-ordenador#:~:text=Por%20lo%20general%2C%20la%20vida,entre%20los%203%20%2D%205%20a%C3%B1os.}}.
\begin{equation}
    \textbf{Coste Anual} = \frac {\text{Coste Materiales}}{\text{5 años}} = \text{precio €/año}
\end{equation}

\begin{equation}
    \textbf{Coste Mensual} = \frac {\text{Coste Anual}}{\text{12 meses}} = \text{precio €/mes}
\end{equation}

\begin{equation}
    \textbf{Materiales x N Meses} = \text{precio mes €/mes} \times \text{N meses} = \text{precio €}
\end{equation}

------------------

\subsection{PRESUPUESTO}

En esta sección se detallará los costos del proyecto. Se considerará el perfil de un puesto junior de desarrollador fullstack,
teniendo en cuenta que el salario anual oscila entre los 17.000€ y 22.000€ \footnote{\url{https://www.glassdoor.es/Sueldos/desarrollador-full-stack-junior-sueldo-SRCH_KO0,31.htm}}.
Teniendo en cuenta que la implementación ha tomado un aproximado de 370 horas, se puede calcular un coste estimado del salario de la desarroladora, el cual se detalla a continuación:

\begin{equation}
    \textbf{Salario medio fullstack junior} =  \frac {\text{22.000 €/Año} }{ \text{12 meses}} = \text{precio mes €/Mes}
\end{equation}

El salario por hora se calcula dividiendo el salario mensual entre 160 horas, que es el promedio de horas laborales al mes:

\begin{equation}
    \textbf{Salario por Hora} = \frac {\text{precio mes €/Mes}}{160 \text{ Horas}} = \text{precio hora €/Hora}
\end{equation}

Teniendo en cuenta las 370 horas de trabajo, se puede calcular el coste total del salario:

\begin{equation}
    \textbf{Coste Salario} = \text{370 Horas} \times \text{precio hora €/Hora} = \text{salario €}
\end{equation}

Para el diseño, se tendrá en cuenta el perfil de un diseñador gráfico, cuyo salario anual oscila entre los 17.000 € - 22.000 € \footnote{\url{https://www.glassdoor.es/Sueldos/disenador-grafico-sueldo-SRCH_KO0,17.htm}}. Teniendo en cuenta que el diseño ha supuesto un aproximado de 16 horas. Partiendo de esta información, se puede calcular el coste total del salario del diseñador, el cual se detalla a continuación:

\begin{equation}
    \textbf{Salario medio Diseñador Gráfico} =  \frac {\text{19.500 €/Año} }{ \text{12 meses}} = \text{1.625 €/Mes}
\end{equation}

El salario por hora se calcula dividiendo el salario mensual entre 160 horas, que es el promedio de horas laborales al mes:

\begin{equation}
    \textbf{Salario por Hora} = \frac {\text{1.625 €/Mes}}{160 \text{ Horas}} = \text{10,16 €/Hora}
\end{equation}

Teniendo en cuenta las X horas de trabajo, se puede calcular el coste total del salario del diseñador:

\begin{equation}
    \textbf{Coste Salario} = \text{X Horas} \times \text{10,16 €/Hora} = \text{X €}
\end{equation}

Sumando los costes de salario y materiales, se obtiene el coste total del proyecto (Figura \ref{tab:coste_total}).

\begin{table}[H]
    \centering
    \begin{tabular}{|c|c|ll}
    \cline{1-2}
    \multicolumn{1}{|l|}{Concepto} & \multicolumn{1}{l|}{Coste (€)} &  &  \\ \cline{1-2}
    Salario Desarrollador           & 2.500,00                        &  &  \\ \cline{1-2}
    Salario Diseñador               & 169,60                         &  &  \\ \cline{1-2}
    Materiales                      & 113,04                       &  &  \\ \cline{1-2}
    \textbf{Total}                   & \textbf{2.782,64}               &  &  \\ \cline{1-2}
    \end{tabular}
    \caption{Coste Total del Proyecto.}
    \label{tab:coste_total}
\end{table}



	% Desarrollo bajo sprints: 
	% 	1. Permitir registros y login de usuarios
	% 	2. Desarrollo del sistema de incidencias
	% 	3. Desarrollo del sistema de denuncias administrativas y accidentes
	% 	4. Desarrollo del sistema de croquis
	%   5. Instalación de la aplicación de manera automática
	% Presupuesto

	\chapter{Diseño}

\section{Enfoque del proyecto}
\subsection{Proyecto como aplicación final}
Como punto de partida inicial se plantea el desarrollo de una aplicación completa y autosuficiente, para su instalación en un dispositivo móvil Android.

Como ventajas de este enfoque el usuario no depende de la disponibilidad del servidor donde se aloja la aplicación, ya que la ésta se instala en su propio dispositivo, además al operar localmente suelen ofrecer un menor tiempo de respuesta. Sin embargo, también presenta desventajas, como el mayor esfuerzo para actualizar la aplicación en cada dispositivo, ya que recae en el usuario o la limitación del entorno, que dificulta la flexibilidad y escalabilidad, integrar nuevas funcionalidades o realizar cambios son tareas más complejas de realizar porque afectan a todo el proyecto.

\subsection{Proyecto como servicio}
Un proyecto como servicio es una aplicación modular y distribuida en la que el backend funciona como un servicio independiente de la plataforma, permitiendo la interacción con varios tipos de clientes (web, móvil, etc.) con conexión a Internet. 

Presenta gran flexibilidad ante la impementación de mejoras, ya que se pueden realizar cambios en el backend sin afectar al frontend, y viceversa. Como puede ser agregar nuevas funcionalidades o módulos (como más servicios de análisis de gastos o incorporar diferentes funcionalidades de ahorro) sin afectar el resto del sistema. A medida que los usuarios crecen, el backend como servicio puede escalarse fácilmente. De esta manera, se adapta bien a un entorno donde puede servir como base para otros clientes o dispositivos, es un sistema más versátil y eficiente. Sin embargo, también presenta desventajas, como dependencia de la red, ya que requiere una conexión estable entre cliente y servidor, lo cual puede ser una limitación si el servicio se utiliza sin darse esa condición. Implementar un proyecto como servicio suele requerir un esfuerzo inicial mayor que la aplicación final para planificar y configurar una arquitectura escalable y segura\cite{galster2014variability}.\\


Dadas las dos opciones anteriores, la que mejor se adapta a este proyecto es el enfoque como servicio, ya que se espera que la aplicación pueda ser utilizada por un número creciente de usuarios y que se puedan implementar mejoras frecuentes y nuevas funcionalidades en el futuro, motivado principalmente por el uso de metodología ágil en el desarollo de la misma. Además permite el acceso multiplataforma, aunque se considera útil la aplicación para dispositivos móviles, no se quiere limitar el alcance de la misma, sino también poder consultar el conjunto de movimientos desde ordenadores, que nos ofrecen mayor comodidad para analizar los datos. De este modo se va a priorizar como aproximación inicial una solución que se pueda usar como mínimo desde cualquiera de estos dos tipos de dispositivos. Se opta por una aplicación web, donde el backend funciona como un servicio independiente y el frontend como cliente.

\begin{figure}[ht!]
    \centering
    \includegraphics[height = 50mm]{imagenes/proyecto_servicio.drawio.png}
    \caption{Proyecto como servicio}
    \label{fig:proyecto_como_servicio}
    \end{figure}


\section{Arquitectura software}
Para definir la arquitectura de software de la aplicación, se analizaron varias opciones disponibles, cada una con características y beneficios diferentes\cite{albin2003art}\cite{garimilla2024art}:

\begin{itemize}
\item Arquitectura Monolítica\\
Las arquitecturas monolíticas agrupan todos los componentes de la aplicación en una única unidad de despliegue, lo cual facilita la implementación y es adecuado para aplicaciones pequeñas o medianas. Sin embargo, este enfoque suele presentar limitaciones en cuanto a flexibilidad y escalabilidad, ya que todos los módulos están estrechamente acoplados. Lo que provoca que las opciones de actualización y expansión se vean reducidas.

\item Arquitectura Basada en Microservicios\\
Una arquitectura de microservicios organiza la aplicación en servicios pequeños e independientes que pueden desarrollarse, desplegarse y escalarse de manera autónoma. Cada microservicio puede implementarse con el lenguaje y las herramientas más adecuadas para su función, lo que aporta flexibilidad y facilita la escalabilidad. Además, su naturaleza distribuida añade seguridad, ya que los servicios se comunican a través de una API y no de manera directa, ofreciendo una capa adicional de protección. Sin embargo, esta estructura aumenta la complejidad de implementación y requiere mayor infraestructura, lo cual puede ser innecesario para aplicaciones con un alcance más específico\cite{RedHat2023}\cite{lopez2017arquitectura}.

\item Arquitectura Cliente-Servidor\\
La arquitectura cliente-servidor divide la aplicación en dos componentes principales: el cliente (interfaz de usuario) y el servidor (lógica de negocio y acceso a datos). La comunicación entre cliente y servidor se puede gestionar mediante una API REST (Representational State Transfer), que utiliza solicitudes HTTP \footnote{HyperText Transfer Protocol: es un protocolo cliente-servidor para comunicar aplicaciones por medio de peticiones de datos y recursos} (como GET, POST, PUT y DELETE) para intercambiar información. Este enfoque proporciona una separación clara entre la interfaz y la lógica de negocio, facilitando el mantenimiento y la escalabilidad sin necesidad de una infraestructura compleja como en el caso de los microservicios.

\end{itemize}

Los enfoques tradicionales con arquitectura monolítica limitan a los desarrolladores en cuanto al lenguaje de programación y entorno de ejecución. Por otro lado, las otras dos opciones permiten a los desarrolladores elegir el lenguaje de programación y el entorno de ejecución más adecuado para cada servicio. Además, por su naturaleza distribuida facilitan la escalabilidad y la flexibilidad del sistema, y aportan seguridad al actuar como una primera línea de defensa ante ataques, ya que los servicios se comunican entre sí por ejemplo mediante una API y no directamente. 

La arquitectura basada en microservicios no se justifica para las necesidades actuales del proyecto, que al resolver una necesidad específica no se beneficia de las ventajas que ofrece pero si podría verse afectado su desarrollo por la complejidad que añade. 

En su lugar, se ha optado por implementar una arquitectura cliente-servidor para la aplicación, donde el cliente (generalmente un navegador web) se corresponde con la interfaz de usuario y el servidor gestiona la lógica de negocio y la base de datos. La comunicación entre ambos se realiza mediante una API REST. Esto permite modularidad, escalabilidad, y facilidad de mantenimiento, facilitando una futura migración a microservicios si es necesario, ya que la API REST puede servir como interfaz entre servicios independientes. La API REST en esta arquitectura conecta de forma sencilla y eficiente los datos y funcionalidades del backend con el frontend, pensada como una colección de herramientas escalable e independiente que realizan tareas específicas. 

\section{Herramientas de desarrollo}

\section{Diseño lógico}
Diagramas UML?

\section{Diseño wireframes / mockups / prototipos para el frontend}

	\input{secciones/07_implementacion}

	% \input{secciones/08_costes}

	% Conclusiones
	\chapter{Conclusiones y trabajos futuros}

\section{Objetivos} \label{sect:goals}
\textit{El objetivo principal de este proyecto es desarrollar una aplicación web (\textbf{SIGMA}: Sistema Inteligente de Gestión Monetaria Automatizada) 
para la gestión financiera. Usando tecnologías de reconocimiento
óptico de caracteres (OCR) y geolocalización, permite al usuario llevar a cabo un 
análisis de gastos detallado y facilita una gestión eficiente de su dinero
a través de un monedero virtual, introduciendo los datos de 
forma sencilla.}

Para el cumplimiento de este objetivo general se plantean los siguientes objetivos específicos:
\begin{enumerate}
    \item Analizar las herramientas de gestión financiera existentes en el mercado para su aplicación en el desarrollo del proyecto.  
    \item Plantear un diseño escalable y desarrollar una arquitectura de software modular para facilitar la integración de futuras funcionalidades o mejoras. 
    \item Diseñar e implementar la interfaz de usuario para permitir la creación 
         de transacciones (ingresos, gastos y ahorros), categorías de gasto y objetivos de ahorro personalizables y modificables.
    \item Implementar herramientas de visualización de datos por medio de mapas, gráficos y resúmenes automatizados para analizar los patrones de gasto y realizar un seguimiento claro de los mismos.
    \item Estudiar e integrar tecnología de reconocimiento óptico de caracteres (OCR) y de reconocimiento de patrones en un texto para escanear tickets de compra; como punto de partida al procesamiento del texto obtenido.
    \item Implementar un sistema que permita automatizar la inserción de gastos en la aplicación, extrayendo datos relevantes de los tickets de compra para que la interverción por parte del usuario en el proceso sea mínima.
    \item Facilitar la búsqueda de comercios. Integrar en la aplicación capacidades de geolocalización para identificar en las transacciones el comercio en el que se ha realizado dicho gasto.   
    
\end{enumerate}

\section{Conclusiones}
\section{Conclusiones}
A lo largo del desarrollo de este proyecto, se han alcanzado satisfactoriamente los objetivos establecidos en la fase inicial. Estos objetivos sirvieron como guía para asegurar que cada funcionalidad cumpliera con los requerimientos del usuario y aportara una solución eficaz en el ámbito de la gestión financiera. A continuación, se detallan las conclusiones obtenidas en relación con cada uno de los objetivos planteados:

\begin{itemize}
    \item \textbf{Análisis de herramientas de gestión financiera}: El estudio de las herramientas existentes en el mercado permitió identificar las características y limitaciones más relevantes en el área de la gestión financiera personal. Gracias a este análisis, se establecieron funcionalidades clave para la aplicación, con un enfoque en simplicidad, usabilidad y flexibilidad.

    \item \textbf{Arquitectura modular y escalable}: Se ha diseñado e implementado una arquitectura de software modular, cumpliendo con el objetivo de facilitar la escalabilidad del proyecto. Esto asegura que futuras integraciones, como nuevas funcionalidades o mejoras en el rendimiento, puedan añadirse sin afectar a los componentes existentes. La arquitectura modular permite una mayor adaptabilidad a las necesidades de los usuarios a lo largo del tiempo.

    \item \textbf{Diseño de la interfaz de usuario}: La interfaz permite a los usuarios gestionar sus transacciones de ingresos, gastos y ahorros de manera eficiente. Se implementaron opciones para categorizar gastos y definir objetivos de ahorro personalizables, permitiendo a los usuarios tener un control claro sobre su situación financiera. Se cumple, por tanto, el objetivo de ofrecer una experiencia de usuario intuitiva y de bajo esfuerzo en la creación y modificación de transacciones y categorías.

    \item \textbf{Visualización de datos mediante gráficos, mapas y resúmenes}: El sistema de visualización de datos proporciona gráficos y resúmenes automáticos de gastos, ofreciendo a los usuarios un análisis visual y detallado de sus patrones de consumo. Además, la visualización de gastos en mapas ayuda a los usuarios a relacionar sus gastos con ubicaciones geográficas específicas, lo cual contribuye a una mejor comprensión y seguimiento de los hábitos de consumo.

    \item \textbf{Integración de tecnología OCR y procesamiento de tickets}: La integración de la tecnología OCR permite escanear tickets de compra y extraer toda la información del ticket de manera eficiente. Esta funcionalidad es el punto de partida hacia un procesamiento automatizado de los datos de los tickets, reduciendo la necesidad de intervención manual por parte del usuario. Este logro representa un avance en la automatización y precisión de la gestión de gastos.

    \item \textbf{Automatización en la inserción de gastos}: Se implementó un sistema de automatización que extrae los datos relevantes de los tickets de compra, lo que minimiza la intervención del usuario en el registro de sus gastos. Esto permite a los usuarios registrar sus gastos de forma rápida y cómoda, mejorando la eficiencia del proceso y cumpliendo así con el objetivo de facilitar la inserción de datos en la aplicación.

    \item \textbf{Geolocalización y búsqueda de comercios}: La integración de la funcionalidad de geolocalización permite identificar y registrar el comercio asociado a cada transacción en función de la ubicación. Esto facilita a los usuarios la identificación de sus transacciones por ubicación y les brinda información adicional a localizar sus gastos en los diferentes comercios. Esta funcionalidad contribuye a una mayor precisión y detalle en el análisis de gastos.

\end{itemize}

Por tanto, este proyecto ha logrado desarrollar una solución completa y eficiente en el ámbito de la gestión financiera personal, aportando una herramienta que integra tecnologías avanzadas y proporciona una experiencia de usuario optimizada. Esta aplicación representa una contribución al estado del arte en la gestión de finanzas personales, ya que combina funciones de análisis de datos, automatización y geolocalización, adaptadas a las necesidades de los usuarios.

Durante el desarrollo de este proyecto, he tenido la oportunidad de enriquecerme en múltiples aspectos. Crear esta solución desde cero y de manera autónoma me ha permitido mejorar significativamente mis habilidades de análisis, ya que fue necesario plantear una construcción completa en el inicio que tuviera sentido como una unidad, para poder evaluar y determinar con claridad las funcionalidades y restricciones necesarias en cada fase que se implementaría. Por otro lado, en la búsqueda de soluciones y en la implementación de las mismas, he tenido la oportunidad de aprender y aplicar nuevas tecnologías y herramientas, lo que ha ampliado mi conocimiento y experiencia en el desarrollo de aplicaciones web.

Además, fue particularmente gratificante encontrar una solución eficaz para una problemática que enfrentaba personalmente y descubrir que esta también resultaba útil para personas de mi entorno. Compartir y discutir los primeros planteamientos sobre la aplicación con otras personas me ayudó a refinar y enfocar las necesidades del proyecto, logrando una mejor comprensión de sus objetivos y una implementación más alineada con las expectativas de los usuarios.

Este proyecto representó un reto de principio a fin, que finalmente ha cumplido con las expectativas y me ha permitido adquirir una valiosa experiencia en el desarrollo de software, desde la concepción de la idea hasta la implementación de una solución funcional y eficiente. En resumen, esta experiencia me ha dejado no solo con nuevas competencias técnicas, sino también con una comprensión más amplia del proceso de desarrollo y de la utilidad de una solución bien diseñada desde el primer momento.


\section{Trabajos futuros}
...

En la actualidad, cada vez son más los comercios que ofrecen la posibilidad de recibir 
el ticket en formato digital. En un futuro probablemente todos los tickets serán 
digitales para reducir el impacto ambiental del uso del papel y abaratar costes.

Por lo que se incluirá en la aplicación la opción de escaneo de tickets digitales.


Versión gratuita no almacena tickets.
Versión de pago para amortizar el almacenaje, que sí permita almacenar todos los tickets.

	% Trabajos futuros
	
	\newpage
	\bibliography{bibliografia}
	\bibliographystyle{plain}
	
\end{document}
