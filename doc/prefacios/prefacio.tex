\thispagestyle{empty}

\begin{center}
{\large\bfseries SIGMA \\ Sistema Inteligente de Gestión Monetaria Automatizada }\\
\end{center}
\begin{center}
Elena Ortega Contreras\\
\end{center}

%\vspace{0.7cm}

\vspace{0.5cm}
\noindent\textbf{Palabras clave}: \textit{software libre, fintech, gestión financiera, Reconocimiento Óptico de Caracteres, Geolocalización, diseño de interfaz de usuario, análisis, desarrollo ágil, cliente-servidor, web}
\vspace{0.7cm}

\noindent\textbf{Resumen}\\

En una época en la que los precios son muy elevados, pero a la vez el consumo parece estar desbocado, es importante administrar el dinero de manera responsable. Esto implica planificación y control sobre cómo y dónde se gasta. Sin embargo, no es fácil hacer el seguimiento de las finanzas personales, ya que se juntan diversos factores a tener en cuenta: tener varias fuentes de ingresos y/o gastos, planes de ahorro, diferentes cuentas bancarias, uso de dinero tanto efectivo como digital, etc. por lo que en ocasiones resulta tedioso llevar las cuentas al día para analizar correctamente el consumo.

Ante la diversidad de métodos de pago mencionada y la necesidad de una gestión eficiente de las finanzas, este proyecto propone unificar la información de todas las fuentes de ingresos y gastos del usuario. Al ofrecer una visión global y realista de su situación económica, la aplicación permitirá a los usuarios tener un control centralizado sobre sus finanzas, facilitando la toma de decisiones informadas y mejorando su capacidad para planificar, anticiparse y gestionar sus recursos financieros de manera más efectiva.

Se centra en el control del dinero analizando el consumo mes a mes y en el ahorro guiado por la planificación, anticipación y previsión de gastos. Para ello se introduce el concepto de un monedero virtual, que cada mes empieza de cero, se va llenando con los ingresos y se vacía con los gastos y los ahorros dedicados a determinados objetivos. 

\cleardoublepage

\begin{center}
{\large\bfseries Automated Intelligent Monetary Management System}\\
\end{center}
\begin{center}
	Elena Ortega Contreras\\
\end{center}
\vspace{0.5cm}
\noindent\textbf{Keywords}: \textit{open source, fintech, financial management, optical character recognition, geolocation, user interface design, analytics, agile development, client-server, web}
\vspace{0.7cm}

\noindent\textbf{Abstract}\\
At a time when prices are very high, but at the same time consumption seems to be rampant, it is important to manage money responsibly. This involves planning and control over how and where you spend. However, it is not easy to keep track of personal finances, as there are a number of factors to consider. several factors to take into account: having several sources of income and/or expenses, savings plans, different accounts and savings plans, different bank accounts, use of both cash and digital money, etc., so it is not easy to cash and digital money, etc., which sometimes makes it tedious to keep up to date to keep the accounts up to date in order to correctly analyze consumption.

Given the diversity of payment methods mentioned above and the need for efficient financial management, this project proposes to unify information from all sources of income and expenses of the user. By providing a global and realistic view of their financial situation, the application will enable users to have centralized control over their finances, facilitating informed decision-making and improving their ability to plan, anticipate and manage their financial resources more effectively.

It focuses on money control by analyzing consumption month by month and on savings guided by planning, anticipating and forecasting expenses. To this end, the concept of a virtual purse is introduced, which each month starts from zero, is filled with income and emptied with expenses and savings dedicated to certain objectives. 

\cleardoublepage

\thispagestyle{empty}

\noindent\rule[-1ex]{\textwidth}{2pt}\\[4.5ex]

D. \textbf{José Manuel Soto Hidalgo}, Profesor Titular del departamento de Ingeniería de Computadores, Automática y Robótica

\vspace{0.5cm}

\textbf{Informo:}

\vspace{0.5cm}

Que el presente trabajo, titulado \textit{\textbf{SIGMA: Sistema Inteligente de Gestión Monetaria Automatizada}},
ha sido realizado bajo mi supervisión por \textbf{Elena Ortega Contreras}, y autorizo la defensa de dicho trabajo ante el tribunal
que corresponda.

\vspace{0.5cm}

Y para que conste, expiden y firman el presente informe en Granada a noviembre de 2024.

\vspace{1cm}

\textbf{El/la director(a)/es: }

\vspace{5cm}

\noindent \textbf{José Manuel Soto Hidalgo}

\chapter*{Agradecimientos}

A mi familia, mis amigos y mi tutor, por apoyarme siempre y motivarme a conseguir mis metas.

